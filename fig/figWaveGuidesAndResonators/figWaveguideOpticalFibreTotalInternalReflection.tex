\documentclass{standalone}
\usepackage{fontspec}
\usepackage{tikz}
\usepackage{pgfplots}
\pgfplotsset{compat=1.9}
\usepackage[europeanresistors]{circuitikz}
\usetikzlibrary{intersections,decorations.markings,calc,decorations.pathreplacing}
\usepackage{polyglossia}                %is loaded the last
\usepackage{siunitx}
\usepackage{amsmath}
\newcommand*{\kvec}[1]{{\ensuremath{{\boldsymbol{#1}}}}}
\newcommand*{\kvecsub}[2]{{\ensuremath{{\boldsymbol{#1}}}_{\textup{#2}}}}

\newcommand*{\ax}{\ensuremath{{\boldsymbol{a}}_{\textup{x}}}}
\newcommand*{\ay}{\ensuremath{{\boldsymbol{a}}_{\textup{y}}}}
\newcommand*{\az}{\ensuremath{{\boldsymbol{a}}_{\textup{z}}}}
%
\newcommand*{\arho}{\ensuremath{{\boldsymbol{a}}_{\rho}}}
\newcommand*{\aphi}{\ensuremath{{\boldsymbol{a}}_{\phi}}}
%
\newcommand*{\ar}{\ensuremath{{\boldsymbol{a}}_{\textup{r}}}}
\newcommand*{\atheta}{\ensuremath{{\boldsymbol{a}}_{\theta}}}

\newcommand*{\aN}{\ensuremath{{\boldsymbol{a}}_N}}
\newcommand*{\aR}{\ensuremath{{\boldsymbol{a}}_{\textup{R}}}}

\newcommand*{\au}{\ensuremath{{\boldsymbol{a}}_u}}
\newcommand*{\av}{\ensuremath{{\boldsymbol{a}}_v}}
\newcommand*{\aw}{\ensuremath{{\boldsymbol{a}}_w}}

\newcommand*{\Ex}{\ensuremath{{\boldsymbol{E}}_x}}
\newcommand*{\Ey}{\ensuremath{{\boldsymbol{E}}_y}}
\newcommand*{\Ez}{\ensuremath{{\boldsymbol{E}}_z}}
%
\newcommand*{\Erho}{\ensuremath{{\boldsymbol{E}}_{\rho}}}
\newcommand*{\Ephi}{\ensuremath{{\boldsymbol{E}}_{\phi}}}
%
\newcommand*{\Er}{\ensuremath{{\boldsymbol{E}}_r}}
\newcommand*{\Etheta}{\ensuremath{{\boldsymbol{E}}_{\theta}}}
%===========================
\newcommand{\RightAngle}[4][5pt]{%
        \draw[gray] ($#3!#1!#2$)
        --($ #3!2!($($#3!#1!#2$)!.5!($#3!#1!#4$)$) $)
        --($#3!#1!#4$) ;        }


\setmainlanguage[numerals=maghrib]{arabic}     %for english numbers use numerals=maghrib, for arabic numerals=arabicdigits
\setotherlanguages{english}


\newfontfamily\arabicfont[Scale=1.0,Script=Arabic]{Scheherazade}
%\newfontfamily\urdufont[Scale=1.0,Script=Arabic]{XB Tabriz}
\newfontfamily\urdufont[Scale=1.25,WordSpace=6.0,,Script=Arabic]{Jameel Noori Nastaleeq}

%draws left and right arrows where needed e.g.  
% \draw[->-=0.5] (0,0)--(3,0); draws arrow at the middle
\tikzset{->-/.style={decoration={markings, mark=at position #1 with {\arrow{latex}}},postaction={decorate}}}
\tikzset{-<-/.style={decoration={markings, mark=at position #1 with {\arrow{latex reversed}}},postaction={decorate}}}


\pgfmathsetmacro{\cX}{2} 
\pgfmathsetmacro{\cY}{0.6} 
\pgfmathsetmacro{\axon}{0.5} 
\pgfmathsetmacro{\dendrite}{0.2} 

\def\coneEye at (#1,#2){(#1,#2) to [out=0,in=-160] ++(\cX,1/3*\cY) --++(0,1/3*\cY) to [out=160,in=0] ++(-\cX,1/3*\cY) .. controls (#1-5/3*\cY,#2+1.5*\cY) and (#1-5/3*\cY,#2-0.5*\cY) .. (#1,#2);
%\path(#1,#2+\cY) .. controls (#1-5/3*\cY,#2+1.5*\cY) and (#1-5/3*\cY,#2-0.5*\cY) .. (#1,#2)coordinate[pos=0.5](coneOut);
%\draw(coneOut)--++(-3*\axon,0); 
}
%
\begin{document}
\begin{urdufont}
\begin{tikzpicture}
%
\pgfmathsetmacro{\x}{0.3} 
\pgfmathsetmacro{\y}{1} 

\pgfmathsetmacro{\kb}{0.5} 
\pgfmathsetmacro{\xb}{\kb*\x} 
\pgfmathsetmacro{\yb}{\kb*\y} 

\pgfmathsetmacro{\kc}{0.3} 
\pgfmathsetmacro{\xc}{\kc*\x} 
\pgfmathsetmacro{\yc}{\kc*\y} 

\pgfmathsetmacro{\la}{1.5} 
\pgfmathsetmacro{\lb}{1} 
\pgfmathsetmacro{\lc}{0.75}

\pgfmathsetmacro{\ll}{\la+\lb+\lc+2}
\pgfmathsetmacro{\a}{30}
\pgfmathsetmacro{\lll}{\yb/sin(\a)}

\draw (0,-\y)--++(\ll,0);
\draw (0,\y)--++(\ll,0);
\draw (0,-\yb)--++(\ll,0);
\draw (0,\yb)--++(\ll,0);
\draw (0,-\y)--(0,\y);
\draw[name path=rightEdge] (\ll,-\y)--(\ll,\y);
\draw[gray,dashed](-1,0)--(\ll+1,0)node[right]{محور};
%ray
\draw[-<-=0.75] (0,0)--++(-130:1);
\draw[->-=0.65](0,0)--++(30:\lll)coordinate(ra)coordinate(angleA);
\draw[->-=0.5](ra)--++(-30:2*\lll)coordinate(ra);
\draw[->-=0.5](ra)--++(30:2*\lll)coordinate(ra);
\path[name path=exitRay](ra)--++(-30:2*\lll)coordinate(rb);
\draw[->-=0.5,name intersections={of={exitRay and rightEdge}}](ra)--(intersection-1);
\draw[->-=0.5] (intersection-1)--++(-45:1);
%text
\draw (\ll/2,\yb+0.25) node[]{$n_2$};
\draw (\ll/2,-\yb-0.25) node[]{$n_2$};
\draw  (\ll/2,\yb-0.25)  node[]{$n_1$};
\draw (3/4*\ll,-\yb-0.25)node[]{\RL{غلاف}};
\draw (3/4*\ll,\yb+0.25)node[]{\RL{غلاف}};
\draw (\ll-0.5,-0.1)node[left]{\RL{ریشہ}};
%angles
\draw([shift={(-180:0.4)}]0,0) arc (-180:-130:0.4);
\draw(180+25:0.7)node[]{$\theta_b$};
\draw ([shift={(0:0.3)}]0,0) arc (0:30:0.3);
\draw (15:0.3) to [out=30,in=180]++(0.5,-0.5)node[right]{$\theta_t$};
\draw[gray,dashed](angleA)--++(0,-0.4);
\draw([shift={(-90:0.2)}]angleA) arc (-90:-150:0.2);
\draw (angleA)++(-120:0.4) node{$\theta_i$};
\end{tikzpicture}%
\end{urdufont}
\end{document} 
