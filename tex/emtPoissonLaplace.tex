\باب{پوئسن اور لاپلاس مساوات}
گاوس کے قانون کی نقطہ شکل
\begin{align}
\nabla \cdot \kvec{D}=\rho_h
\end{align}
میں \عددیء{\kvec{D}=\epsilon \kvec{E}} اور حاصل جواب میں \عددیء{\kvec{E}=-\nabla V} پر کرنے سے
\begin{align*}
\nabla \cdot (\epsilon \kvec{E})=-\nabla \cdot (\epsilon \nabla V)=\rho_h 
\end{align*}
یعنی
\begin{align}\label{مساوات_پوئسن_نقطہ}
\nabla \cdot \nabla V=-\frac{\rho_h}{\epsilon}
\end{align}
حاصل ہوتا ہے  جہاں ہر طرف یکساں\فرہنگ{یکساں!ہر طرف}\حاشیہب{homogeneous}\فرہنگ{homogeneous} خاصیت کے خطے میں \عددیء{\epsilon} اٹل قیمت رکھتا ہے۔مساوات \حوالہ{مساوات_پوئسن_نقطہ} \اصطلاح{پوئسن}\فرہنگ{پوئسن مساوات}\حاشیہب{Poisson equation}\فرہنگ{Poisson equation} مساوات  کہلاتا ہے۔

آئیں کارتیسی محدد میں پوئسن مساوات کی شکل حاصل کریں۔یاد رہے کہ کسی بھی متغیرہ \عددیء{\kvec{A}=A_x\ax+A_y\ay+A_z\az} کے لئے
\begin{align*}
\nabla \cdot \kvec{A}=\frac{\partial A_x}{\partial x}+\frac{\partial A_y}{\partial y}+\frac{\partial A_z}{\partial z}
\end{align*}  
کے برابر ہوتا ہے۔اب چونکہ
\begin{align*}
\nabla{V}=\frac{\partial V}{\partial x}\ax+\frac{\partial V}{\partial y}\ay+\frac{\partial V}{\partial z}\az
\end{align*}
 کے برابر ہے لہٰذا
\begin{gather}
\begin{aligned}
\nabla \cdot \nabla V&=\frac{\partial }{\partial x}\left(\frac{\partial V}{\partial x}\right)+\frac{\partial }{\partial y}\left(\frac{\partial V}{\partial y}\right)+\frac{\partial }{\partial z}\left(\frac{\partial V}{\partial z}\right)\\
&=\frac{\partial^2 V}{\partial x^2}+\frac{\partial^2 V}{\partial y^2}+\frac{\partial^2 V}{\partial z^2}
\end{aligned}
\end{gather}
 ہو گا۔

عموماً \عددیء{\nabla \cdot \nabla} کو \عددیء{\nabla^2} لکھا جاتا ہے۔اس طرح پوئسن مساوات کی کارتیسی شکل
\begin{align}\label{مساوات_لاپلاس_پوئسن_کارتیسی_شکل}
\nabla^2 V=\frac{\partial^2 V}{\partial x^2}+\frac{\partial^2 V}{\partial y^2}+\frac{\partial^2 V}{\partial z^2}=-\frac{\rho_h}{\epsilon}
\end{align}
حاصل ہوتی ہے۔ 

حجمی چارج کثافت کی غیر موجودگی، یعنی \عددیء{\rho_h =0} کی صورت میں مساوات \حوالہ{مساوات_پوئسن_نقطہ}
\begin{align}\label{مساوات_لاپلاس_لاپلاس_نقطہ_شکل}
\nabla^2 V=0
\end{align}
صورت اختیار کر لے گی جسے \اصطلاح{لاپلاس}\فرہنگ{لاپلاس مساوات}\حاشیہب{Laplace equation}\فرہنگ{Laplace equation} مساوات کہتے ہیں۔جس حجم کے لئے لاپلاس کی مساوات لکھی گئی ہو اس حجم میں حجمی چارج کثافت صفر ہوتا ہے البتہ اس حجم کی سرحد پر نقطہ چارج یا سطحی چارج کثافت پائی جا سکتیں ہیں۔عموماً سطح پر موجود چارج سے حجم میں پیدا میدان ہی حاصل کرنا مطلوب ہوتا ہے۔کارتیسی محدد میں لاپلاس کی مساوات
\begin{align}\label{مساوات_لاپلاس_لاپلاس_کارتیسی_شکل}
\nabla^2 V=\frac{\partial^2 V}{\partial x^2}+\frac{\partial^2 V}{\partial y^2}+\frac{\partial^2 V}{\partial z^2}=0
\end{align}
صورت رکھتی ہے۔\عددیء{\nabla^2} کو لاپلاسی عامل\فرہنگ{لاپلاسی عامل}\حاشیہب{Laplacian operator}\فرہنگ{Laplacian operator} کہا جاتا ہے۔

لاپلاس مساوات کہتا ہے کہ کسی بھی چارج سے خالی حجم میں ہر صورت \عددیء{\nabla^2=0} ہو گا۔حجم کی شکل کچھ بھی ہو سکتی ہے اور اس کے سرحد پر کسی بھی قسم کا چارج ہو سکتا ہے۔یہ ایک دلچسپ حقیقت ہے۔حجم کے سرحد پر عموماً ایک یا ایک سے زیادہ موصل سطحیں ہوتی ہیں جن پر برقی دباو \عددیء{V_0}، \عددیء{V_1}، \عددیء{V_2} وغیرہ پایا جاتا ہے اور حجم کے اندر میدان کا حصول درکار ہوتا ہے۔کبھی کبھار موصل سطح پر چارج یا \عددیء{\kvec{E}} معلوم ہو گا جس سے حجم کے اندر میدان درکار ہو گا۔اسی طرح کبھی کبھار سرحد پر ایک جگہ چارج اور اس پر دوسری جگہ برقی دباو اور اس پر تیسرے  جگہ عمودی بہاو دیا گیا ہو گا جبکہ حجم کے اندر کے متغیرات درکار ہوں گے۔اس کے برعکس ایسا بھی ممکن ہے کہ حجم میں میدان یا برقی دباو معلوم ہو اور ان معلومات سے سرحد پر چارج یا بہاو یا برقی دباو حاصل کرنا ضروری ہو گا۔

ہم نے لاپلاس کی مساوات برقی دباو کے لئے حاصل کی۔دیکھا یہ گیا ہے کہ انجینئری کے دیگر شعبوں میں کئی متغیرات لاپلاس کے مساوات پر پورا اترتے ہیں۔یہ مساوات حقیقی اہمیت کا حامل ہے۔ 

اس باب میں ہم ایسی کئی مثالیں دیکھیں گے لیکن پہلے یہ حقیقت جاننا ضروری ہے کہ مساوات \حوالہ{مساوات_لاپلاس_لاپلاس_کارتیسی_شکل} کا کوئی بھی جواب ان تمام اقسام کے سرحدی معلومات کے لئے درست ہو گا۔یہ انتہائی تشویشناک بات ہو گی اگر دو مختلف طریقوں سے لاپلاس مساوات کے جوابات حاصل کرنے کے بعد معلوم ہو کہ ان میں سے ایک ٹھیک اور دوسرا غلط جواب ہے۔آئیں اس حقیقت کا ثبوت دیکھیں کہ کسی بھی سرحدی حقائق کو مد نظر رکھتے ہوئے لاپلاس مساوات کا صرف اور صرف ایک ہی جواب حاصل ہوتا ہے۔

\حصہ{مسئلہ یکتائی}
تصور کریں کہ ہم دو مختلف طریقوں سے لاپلاس مساوات کے دو جوابات \عددیء{V_1} اور \عددیء{V_2} حاصل کرتے ہیں۔یہ دونوں جوابات لاپلاس مساوات پر پورا اترتے ہیں لہٰذا
\begin{align*}
\nabla^2 V_1&=0\\
\nabla^2 V_2&=0
\end{align*} 
لکھا جا سکتا ہے جس سے
\begin{align}\label{مساوات_لاپلاس_دو_جوابات_الف}
\nabla^2 (V_1-V_2)=0
\end{align}
حاصل ہوتا ہے۔اب اگر سرحد پر برقی دباو \عددیء{V_s} ہو تب دونوں جوابات سرحد پر یہی جواب دیں گے یعنی سرحد پر
\begin{align*}
V_{1s}=V_{2s}=V_s
\end{align*}
یا
\begin{align*}
V_{1s}-V_{2s}=0
\end{align*}
ہو گا۔صفحہ \حوالہصفحہ{مساوات_توانائی_توانائی_ضرب_برقی_بہاو_کی_ڈھلان} پر مساوات \حوالہ{مساوات_توانائی_توانائی_ضرب_برقی_بہاو_کی_ڈھلان}
\begin{align*}
\nabla \cdot  (V \kvec{D})=V (\nabla \cdot \kvec{D})+\kvec{D} \cdot (\nabla V)
\end{align*}
کا ذکر کیا گیا جو کسی بھی مقداری \عددیء{V} اور کسی بھی سمتیہ \عددیء{\kvec{D}} کے لئے درست ہے۔موجودہ استعمال کے لئے ہم \عددیء{V_1-V_2} کو مقداری اور \عددیء{\nabla(V_1-V_2)} کو سمتیہ لیتے ہوئے
\begin{align*}
\nabla \cdot  [(V_1-V_2) \nabla(V_1-V_2)]&=(V_1-V_2) [\nabla \cdot \nabla(V_1-V_2)]+\nabla(V_1-V_2) \cdot \nabla (V_1-V_2)\\
&=(V_1-V_2) [\nabla^2(V_1-V_2)]+[\nabla(V_1-V_2)]^2
\end{align*}
حاصل ہوتا ہے جس کا تکمل پورے حجم کے لئے
\begin{align}\label{مساوات_لاپلاس_یکتائی_ثبوت_الف}
\int\limits_{\textrm{حجم}} \nabla \cdot  [(V_1-V_2) \nabla(V_1-V_2)] \dif h=\int\limits_{\textrm{حجم}}(V_1-V_2) [\nabla^2(V_1-V_2)] \dif h+\int\limits_{\textrm{حجم}}[\nabla(V_1-V_2)]^2 \dif h
\end{align}
ہو گا۔صفحہ \حوالہصفحہ{مساوات_گاوس_مسئلہ_پھیلاو_تکمل_شکل} پر مساوات \حوالہ{مساوات_گاوس_مسئلہ_پھیلاو_تکمل_شکل} مسئلہ پھیلاو بیان کرتا ہے جس کے مطابق کسی بھی حجمی تکمل کو  بند سطحی تکمل میں تبدیل کیا جا سکتا ہے جہاں حجم کی سطح پر سطحی تکمل حاصل کیا جاتا ہے۔یوں مندرجہ بالا مساوات کے بائیں ہاتھ کو سطحی تکمل میں تبدیل کرتے ہوئے
\begin{align*}
\int\limits_{\textrm{حجم}} \nabla \cdot  [(V_1-V_2) \nabla(V_1-V_2)] \dif h=\oint\limits_{\textrm{سطح}} [(V_{1s}-V_{2s}) \nabla(V_{1s}-V_{2s})] \cdot \dif \kvec{S}=0
\end{align*}
حاصل ہوتا ہے جہاں سرحدی سطح پر \عددیء{V_{1s}=V_{2s}} ہونے کی بنا پر \عددیء{V_{1s}-V_{2s}=0} ہے اور صفر کا تکمل صفر ہی ہوتا ہے۔مساوات \حوالہ{مساوات_لاپلاس_یکتائی_ثبوت_الف} میں دائیں ہاتھ پہلے جزو میں مساوات \حوالہ{مساوات_لاپلاس_دو_جوابات_الف} کے تحت \عددیء{\nabla^2(V_1-V_2)=0} ہے اور صفر کا تکمل صفر ہی ہوتا ہے۔اس طرح مساوات \حوالہ{مساوات_لاپلاس_یکتائی_ثبوت_الف} سے
\begin{align*}
\int\limits_{\textrm{حجم}}[\nabla(V_1-V_2)]^2 \dif h=0
\end{align*}
حاصل ہوتا ہے۔

کسی بھی تکمل کا جواب صرف دو صورتوں میں صفر کے برابر ہو سکتا ہے۔پہلی صورت یہ ہے کہ کچھ خطے میں تکمل کی قیمت مثبت اور کچھ خطے میں اس کی قیمت منفی ہو۔اگر مثبت اور منفی حصے بالکل برابر ہوں تب تکمل صفر کے برابر ہو گا۔موجودہ صورت میں \عددیء{[\nabla(V_1-V_2)]^2} کا تکمل لیا جا رہے ہے اور کسی بھی متغیر کا مربع کسی صورت منفی نہیں ہو سکتا لہٰذا موجودہ تکمل میں ایسا ممکن نہیں ہے۔تکمل صفر ہونے کی دوسری صورت یہ ہے کہ صفر کا تکمل حاصل کیا جا رہا ہو لہٰذا
\begin{align*}
[\nabla(V_1-V_2)]^2 =0
\end{align*}
ہی ہو گا یعنی
\begin{align*}
\nabla (V_1-V_2)=0
\end{align*}
کے برابر ہے۔

اب \عددیء{\nabla (V_1-V_2)=0} کا مطلب ہے کہ \عددیء{V_1-V_2} کی ڈھلان ہر صورت صفر کے برابر ہے۔یہ تب ہی ممکن ہے جب \عددیء{V_1-V_2} کی قیمت کسی بھی محدد کے ساتھ تبدیل نہ ہو یعنی اگر تکمل کے پورے خطے میں
\begin{align*}
V_1-V_2=\textrm{اٹل قیمت}
\end{align*}
ہو۔حجم کے سرحد پر بھی یہ درست ہو گا۔مگر سرحد پر
\begin{align*}
V_1-V_2=V_{1s}-V_{2s}=0
\end{align*}
کے برابر ہے لہٰذا یہ اٹل قیمت ازخود صفر ہے۔یوں
\begin{align}
V_1=V_2
\end{align}
ہو گا۔اس کا مطلب  ہے کہ دونوں جوابات بالکل برابر ہیں۔

مسئلہ یکتائی کو پوئسن مساوات کے لئے بھی بالکل اسی طرح ثابت کیا جا سکتا ہے۔پوئسن مساوات کے دو جوابات \عددیء{V_1} اور \عددیء{V_2} پوئسن مساوات پر پورا اتریں گے لہٰذا \عددیء{\nabla^2 V_1=-\tfrac{\rho_h}{\epsilon}} اور \عددیء{\nabla^2 V_2=-\tfrac{\rho_h}{\epsilon}} لکھے جا سکتے ہیں جن سے \عددیء{\nabla^2(V_1-V_2)=0} حاصل ہوتا ہے۔سرحد پر اب بھی \عددیء{{V_{1s}-V_{2s}= 0}} ہو گا۔یہاں سے آگے ثبوت بالکل یکتائی لاپلاس کی ثبوت کی طرح ہے۔

مسئلہ یکتائی کے تحت سرحدی حقائق کے لئے حاصل کئے  گئے پوئسن یا لاپلاس مساوات کے جوابات ہو صورت برابر ہوں گے۔یہ ممکن نہیں کہ دو مختلف جوابات حاصل کئے جائیں۔ 

\حصہ{نلکی اور کروی محدد میں لاپلاس کی مساوات} 
نلکی محدد میں ڈھلان کی مساوات صفحہ \حوالہصفحہ{مساوات_توانائی_ڈھلان_نلکی} پر مساوات \حوالہ{مساوات_توانائی_ڈھلان_نلکی} دیتا ہے جس سے 
\begin{gather}
\begin{aligned}\label{مساوات_لاپلاس_نلکی_پھیلاو_الف}
\nabla V &= \frac{\partial V}{\partial \rho} \arho+\frac{1}{\rho}\frac{\partial V}{\partial \phi}  \aphi+\frac{\partial V}{\partial z}\az\\
&=-E_\rho \arho-E_\phi \aphi-E_z \az 
\end{aligned}
\end{gather}
لکھتے  ہیں جہاں \عددیء{\kvec{E}=-\nabla V} کا استعمال کیا گیا۔نلکی محدد میں پھیلاو کی مساوات صفحہ \حوالہصفحہ{مساوات_گاوس_نلکی_عمومی_پھیلاو} پر مساوات \حوالہ{مساوات_گاوس_نلکی_عمومی_پھیلاو} دیتا ہے۔اسی مساوات کو سمتیہ \عددیء{\kvec{E}} کے لئے
 \begin{align*}
\nabla \cdot \kvec{E}=\frac{1}{\rho}\frac{\partial (\rho E_{\rho})}{\partial \rho}+\frac{1}{\rho}\frac{\partial E_{\phi}}{\partial \phi}  +  \frac{\partial E_{z}}{\partial z}
\end{align*}
لکھتے ہیں۔اس میں بائیں ہاتھ \عددیء{\kvec{E}=-\nabla V}  اور دائیں ہاتھ مساوات \حوالہ{مساوات_لاپلاس_نلکی_پھیلاو_الف} سے قیمتیں پر کرتے ہوئے
\begin{align*}
\nabla \cdot \nabla V=\frac{1}{\rho}\frac{\partial }{\partial \rho}\left(\rho \frac{\partial V}{\partial \rho}\right)
+\frac{1}{\rho}\frac{\partial }{\partial \phi}\left(\frac{1}{\rho}\frac{\partial V}{\partial \phi}  \right) 
+  \frac{\partial}{\partial z} \left(\frac{\partial V}{\partial z} \right)
\end{align*}
حاصل ہوتا ہے جہاں دونوں جانب منفی علامت کٹ جاتے ہیں۔اس کو یوں
\begin{align}
\nabla^2 V=\frac{1}{\rho}\frac{\partial }{\partial \rho}\left(\rho \frac{\partial V}{\partial \rho}\right)
+\frac{1}{\rho^2}\left(\frac{\partial^2 V}{\partial \phi^2}  \right) 
+  \frac{\partial^2 V}{\partial z^2}\quad {\textrm{نلکی}}
\end{align}
 لکھا جا سکتا ہے جو نلکی محدد میں لاپلاسی مساوات ہے۔

کروی محدد میں بالکل اسی
\begin{align}\label{مساوات_لاپلاس_کروی_لاپلاسی}
\nabla^2 V=\frac{1}{r^2} \frac{\partial}{\partial r} \left(r^2\frac{\partial V}{\partial r} \right)+\frac{1}{r^2 \sin \theta} \frac{\partial}{\partial \theta} \left(\sin \theta \frac{\partial V}{\partial \theta}  \right)+\frac{1}{r^2 \sin \theta}\frac{\partial^2 V}{\partial \phi^2} \quad {\textrm{کروی}}
\end{align}
جبکہ عمومی محدد میں
\begin{align}\label{مساوات_لاپلاس_عمومی_لاپلاسی}
\nabla^2 V=\frac{1}{k_1 k_2 k_3}\left[\frac{\partial}{\partial u}\left(\frac{k_2 k_3}{k_1}\frac{\partial V}{\partial u} \right)+\frac{\partial}{\partial v}\left(\frac{k_1 k_3}{k_2}\frac{\partial V}{\partial v} \right) +\frac{\partial}{\partial w}\left(\frac{k_1 k_2}{k_3}\frac{\partial V}{\partial w} \right)\right] \quad{\textrm{عمومی}}
\end{align}
حاصل کی جا سکتی ہے۔
%==================

\ابتدا{مشق}
مساوات \حوالہ{مساوات_لاپلاس_کروی_لاپلاسی} حاصل کریں۔
\انتہا{مشق}
%============================

\حصہ{لاپلاس مساوات کے حل}
لاپلاس مساوات حل کرنے کے کئی طریقے ہیں۔سادہ ترین مسئلے، سادہ تکمل سے ہی حل ہو جاتے ہیں۔ہم اسی سادہ تکمل کے طریقے سے کئی مسئلے حل کریں گے۔یہ طریقہ صرف اس صورت قابل استعمال ہوتا ہے جب میدان یک سمتی ہو یعنی جب یہ محدد کے تین سمتوں میں سے صرف ایک سمت میں تبدیل ہوتا ہو۔چونکہ اس کتاب میں محدد کے تین نظام استعمال کئے جا رہے ہیں لہٰذا معلوم ایسا ہوتا ہے کہ  کل نو مسئلے ممکن ہیں۔درحقیقت ایسا نہیں ہے۔کارتیسی محدد میں \عددیء{x} سمت میں تبدیل ہوتے میدان کا حل بالکل ویسا ہی ہے جیسے \عددیء{y} یا \عددیء{z} سمت میں تبدیل ہوتے میدان کا حل۔اسی طرح \عددیء{x} محدد سے کسی زاویے پر سیدھی لکیر کی سمت میں تبدیل ہوتا میدان بھی بالکل اسی طرح حل ہو گا۔یوں کارتیسی محدد میں کسی بھی سمت میں تبدیل ہوتے میدان اور \عددیء{x} سمت میں تبدیل ہوتے میدان کے حل بالکل ایک جیسے ہوں گے لہٰذا کارتیسی محدد میں صرف ایک مسئلہ حل کرنا درکار ہے۔نلکی محدد میں \عددیء{z} محدد کے ساتھ تبدیل ہوتے میدان کو ہم کارتیسی محدد میں دیکھ لیں گے لہٰذا یہاں کل دو مسئلے حل کرنا درکار ہے جبکہ کروی محدد میں بھی دو مسئلے پائے جاتے ہیں۔آئیں ان تمام کو باری باری حل کریں۔
%=================

\ابتدا{مثال}
تصور کریں کہ \عددیء{V} صرف \عددیء{x} محدد کے ساتھ تبدیل ہوتی ہو۔دیکھتے ہیں کہ ایسی صورت میں لاپلاس مساوات کا حل کیا ہو گا۔اس پر بعد میں غور کریں گے کہ حقیقت میں ایسی کون سی صورت ہو گی کہ \عددیء{V} صرف \عددیء{x} محدد کے ساتھ تبدیل ہوتا ہو۔ایسی صورت میں لاپلاس مساوات
\begin{align*}
\frac{\partial^2 V}{\partial x^2}=0
\end{align*}
شکل اختیار کر لے گا۔چونکہ \عددیء{V} کی قیمت صرف \عددیء{x} پر منحصر ہے لہٰذا مندرجہ بالا مساوات کو
\begin{align*}
\frac{\dif{\hspace{0.1pt} ^2} V}{\dif x^2}=0
\end{align*}
لکھا جا سکتا ہے۔پہلی بار تکمل لیتے ہوئے
\begin{align*}
\frac{\dif V}{\dif x}=A
\end{align*}
حاصل ہوتا ہے۔دوبارہ تکمل لیتے ہوئے
\begin{align}\label{مساوات_لاپلاس_کارتیسی_حل}
V=Ax+B
\end{align}
حاصل ہوتا ہے جو لاپلاس مساوات کا حل ہے۔یہ کسی بھی سیدھی لکیر کی سمت میں تبدیل ہوتے برقی دباو کے مسئلے کو ظاہر کرتا ہے جہاں اس لکیر کو \عددیء{x} کہا جائے گا۔\عددیء{A} اور \عددیء{B} دو درجی تکمل کے مستقل ہیں جن کی قیمتیں سرحدی شرائط کی مدد سے حاصل کی جاتی ہیں۔

آئیں مساوات \حوالہ{مساوات_لاپلاس_کارتیسی_حل} کا مطلب سمجھیں۔اس کے مطابق برقی دباو کا دارومدار صرف \عددیء{x} پر ہے  جبکہ \عددیء{y} اور \عددیء{z} کا اس کی قیمت پر کوئی اثر نہیں۔\عددیء{x} کی کسی بھی قیمت پر یعنی \عددیء{x=x_0} سطح پر \عددیء{V} کی قیمت اٹل ہو گی۔ایسی ہم قوہ سطحیں \عددیء{x} محدد کے عمودی ہوں گی۔آپ دیکھ سکتے ہیں کہ مساوات \حوالہ{مساوات_لاپلاس_کارتیسی_حل} یہ متوازی چادر کپیسٹر کا حل ہے۔

ہم ایسے کپیسٹر کے دونوں چادروں پر برقی دباو اور چادروں کا \عددیء{x} محدد پر مقام بیان کرتے ہوئے  \عددیء{A} اور \عددیء{B} کی قیمتیں حاصل کر سکتے ہیں۔یوں اگر کپیسٹر کی پہلی  چادر  \عددیء{x_1} پر ہے جبکہ اس پر برقی دباو \عددیء{V_1} ہے اور اسی طرح دوسری چادر \عددیء{x_2} پر ہے جبکہ اس پر برقی دباو \عددیء{V_2} ہے تب
\begin{align*}
V_1&=A x_1+B\\
V_2&=Ax_2+B
\end{align*}
ہو گا جس سے
\begin{align*}
A&=\frac{V_1-V_2}{x_1-x_2}\\
B&=\frac{V_2 x_1-V_1x_2}{x_1-x_2}
\end{align*}
حاصل ہوتے ہیں۔یوں چادروں کے درمیان
\begin{align}
V=\left(\frac{V_1-V_2}{x_1-x_2}\right)x+\frac{V_2 x_1-V_1x_2}{x_1-x_2}
\end{align}
ہو گا۔

اگر ہم پہلی  چادر کو \عددیء{x=0} اور دوسری چادر کو \عددیء{d} پر تصور کرتے جبکہ اسی ترتیب سے ان کی برقی دباو کو صفر اور \عددیء{V_0} کہتے تب ہمیں
\begin{align}\label{مساوات_لاپلاس_کارتیسی_برقی_دباو}
V=\frac{V_0 x}{d}
\end{align}
حاصل ہوتا جو نسبتاً آسان مساوات ہے۔

باب \حوالہ{باب_کپیسٹر} میں ہم نے سطحی چارج کثافت سے بالترتیب برقی میدان، برقی دباو اور کپیسٹنس حاصل کئے۔موجودہ باب میں ہم پہلے لاپلاس کے مساوات کے حل سے برقی دباو حاصل کرتے ہیں۔برقی دباو سے میدان بذریعہ  \عددیء{\kvec{E}=-\nabla V} اور بہاو بذریعہ \عددیء{\kvec{D}=\epsilon \kvec{E}} حاصل کرتے ہوئے سطحی چارج کثافت حاصل کرتے ہیں جو عمودی بہاو کے برابر ہے۔سطحی چارج کثافت سے سطح پر کل چارج حاصل کرتے ہوئے \عددیء{C=\tfrac{Q}{V}} حاصل کیا جاتا ہے۔ان اقدام کو بالترتیب دوبارہ پیش کرتے ہیں۔
\begin{itemize}
\item
لاپلاس مساوات حل کرتے ہوئے برقی دباو \عددیء{V} حاصل کریں۔
\item
تکمل کے سرحدی شرائط سے تکمل کے مستقل کی قیمتیں حاصل کریں۔
\item
برقی دباو سے برقی میدان اور برقی بہاو  بذریعہ \عددیء{\kvec{E}=-\nabla V} اور \عددیء{\kvec{D}=\epsilon \kvec{E}} حاصل کریں۔
\item
کپیسٹر کے کسی ایک چادر پر برقی بہاو کی قیمت \عددیء{\kvec{D}_S=D_n\aN} حاصل کریں جو سطح کے عمودی ہو گا۔ 
\item
چونکہ سطح پر سطحی چارج کثافت اور عمودی برقی بہاو برابر ہوتے ہیں لہٰذا \عددیء{\rho_S=D_n} ہو گا۔مثبت چارج کثافت کی صورت میں برقی بہاو کا موصل چادر سے اخراج جبکہ منفی چارج کثافت کی صورت میں برقی بہاو کا چادر میں دخول ہو گا۔
\item
سطح پر چارج بذریعہ سطحی تکمل حاصل کریں۔
\item
کپیسٹنس \عددیء{C=\tfrac{Q}{V}} ہو گا۔
\end{itemize}
آئیں ان اقدام کو موجودہ مثال پر لاگو کریں۔

چونکہ موجودہ مثال میں مساوات \حوالہ{مساوات_لاپلاس_کارتیسی_برقی_دباو} کے تحت
\begin{align*}
V=\frac{V_0x}{d}
\end{align*}
ہے لہٰذا
\begin{align*}
\kvec{E}=-\nabla V=-\frac{V_0}{d}\ax
\end{align*}
اور
\begin{align*}
\kvec{D}=-\epsilon \frac{V_0}{d}\ax
\end{align*}
چونکہ بہاو کی سمت مثبت سے منفی چادر کی جانب ہوتی ہے لہٰذا مثبت چادر \عددیء{x=d} پر جبکہ منفی چادر \عددیء{x=0} پر ہے۔مثبت چادر پر
\begin{align*}
\kvec{D}_S=\eval{\kvec{D}}_{x=d}=-\epsilon \frac{V_0}{d}\ax
\end{align*}
کے برابر ہے۔چونکہ مثبت چادر کا
\begin{align*}
\aN=-\ax
\end{align*}
ہے لہٰذا برقی بہاو چادر سے خارج ہو رہا ہے۔یوں
\begin{align*}
\rho_S=\epsilon \frac{V_0}{d}
\end{align*}
ہو گا۔اگر چادر کی سطح کا رقبہ \عددیء{S} ہو تب
\begin{align*}
Q=\int_S \rho_S \dif S=\int \epsilon \frac{V_0}{d} \dif S=\frac{\epsilon V_0 S}{d}
\end{align*}
ہو گا جس سے
\begin{align*}
C=\frac{\epsilon S}{d}
\end{align*}
حاصل ہوتا ہے۔صفحہ \حوالہصفحہ{مساوات_کپیسٹر_دو_چادر_کپیسٹر} پر مساوات \حوالہ{مساوات_کپیسٹر_دو_چادر_کپیسٹر} یہی جواب دیتا ہے۔
\انتہا{مثال}
%=================

اگر مندرجہ بالا مثال میں کپیسٹر کو \عددیء{y} یا \عددیء{z} محدد پر رکھا جاتا تو کپیسٹنس کی قیمت یہی حاصل ہوتی لہٰذا کارتیسی محدد کے لئے ایک مثال حل کر لینا کافی ہے۔نلکی محدد میں \عددیء{z} کے ساتھ تبدیل ہوتے برقی دباو کو حل کرنے سے کوئی نئی بات سامنے نہیں آتی۔یہ بالکل کارتیسی محدد کے مثال کی طرح ہی ہے لہٰذا ہم باری باری \عددیء{\rho} اور \عددیء{\phi} کے ساتھ تبدیل ہوتے برقی دباو کے مسئلے حل کرتے ہیں۔

%========================

\ابتدا{مثال}
اس مثال میں صرف \عددیء{\rho} کے ساتھ تبدیل ہوتے برقی دباو پر غور کرتے ہیں۔ایسی صورت میں لاپلاس کی مساوات
\begin{align*}
\frac{1}{\rho} \frac{\partial}{\partial \rho} \left(\rho \frac{\partial V}{\partial \rho} \right)=0
\end{align*}
یا
\begin{align}\label{مساوات_لاپلاس_ہم_محوری_لاپلاسی}
\frac{1}{\rho}\frac{\dif}{\dif \rho} \left(\rho \frac{\dif V}{\dif \rho} \right)=0
\end{align}
صورت اختیار کر لے گی۔یوں یا
\begin{align*}
\frac{1}{\rho}=0
\end{align*}
ہو گا جس سے
\begin{align}\label{مساوات_لاپلاس_ہم_محوری_لاپلاسی_پہلا-حل}
\rho=\infty
\end{align}
حاصل ہوتا ہے اور یا
\begin{align}\label{مساوات_لاپلاس_ہم_محوری_لاپلاسی_ب}
\frac{\dif}{\dif \rho} \left(\rho \frac{\dif V}{\dif \rho} \right)=0
\end{align}

ہو گا۔اس تفرقی مساوات کو بار بار تکمل لے کر حل کرتے ہیں۔پہلی بار تکمل لیتے ہوئے
\begin{align*}
\rho \frac{\dif V}{\dif \rho}=A
\end{align*}
یا
\begin{align*}
\dif V=A \frac{\dif \rho}{\rho}
\end{align*}
حاصل ہوتا ہے۔دوسری بار تکمل سے
\begin{align*}
V=A \ln \rho+B
\end{align*}
حاصل ہوتا ہے۔یہ ہم قوہ سطحیں نلکی شکل کے ہیں۔یوں یہ مساوات محوری تار کا برقی دباو دیتی ہے۔ہم محوری تار کے بیرونی تار \عددیء{\rho=b} کو برقی زمین اور اندرونی تار \عددیء{\rho=a} کو \عددیء{V_0} برقی دباو پر تصور کرتے ہوئے
\begin{align}\label{مساوات_لاپلاس_ہم_محوری_لاپلاسی_ب_حل}
V=V_0 \frac{\ln \frac{b}{\rho} }{\ln \frac{b}{a} }
\end{align}
حاصل ہوتا ہے۔ہم جانتے ہیں کہ کسی بھی شکل کے چارج سے لامحدود فاصلے پر برقی دباو صفر ہی ہوتا ہے۔اسی وجہ سے ہم لامحدود فاصلے کو ہی برقی زمین کہتے آ رہے ہیں۔یوں لاپلاس مساوات کا پہلا حل یعنی مساوات \حوالہ{مساوات_لاپلاس_ہم_محوری_لاپلاسی_پہلا-حل} ہمارے امید کے عین مطابق ہے۔

مساوات \حوالہ{مساوات_لاپلاس_ہم_محوری_لاپلاسی_ب_حل} کو لے کر آگے بڑھتے ہوئے یوں
\begin{align*}
\kvec{E}=-\nabla V=\frac{V_0}{\rho} \frac{1}{\ln \frac{b}{a}} \arho
\end{align*}
اور
\begin{align*}
D_n&=\eval{D}_{\rho=a}=\frac{\epsilon V_0}{a\ln \frac{b}{a}}\\
Q&=\frac{\epsilon V_0 2 \pi a L}{a \ln \frac{b}{a}}
\end{align*}
حاصل ہوتے ہیں جن سے
\begin{align}
C=\frac{2\pi \epsilon L}{\ln \frac{b}{a}}
\end{align}
حاصل ہوتا ہے۔صفحہ \حوالہصفحہ{مساوات_کپیسٹر_کپیسٹر_ہم_محوری_تار} پر مساوات \حوالہ{مساوات_کپیسٹر_کپیسٹر_ہم_محوری_تار} یہی جواب دیتا ہے۔

مساوات \حوالہ{مساوات_لاپلاس_ہم_محوری_لاپلاسی} کو \عددیء{\rho} سے ضرب دینے سے بھی مساوات \حوالہ{مساوات_لاپلاس_ہم_محوری_لاپلاسی_ب} حاصل ہوتا ہے۔البتہ یہ ضرب صرف اور صرف اس صورت ممکن ہے جب \عددیء{\rho \ne 0} ہو۔یاد رہے کہ \عددیء{\rho=0} کی صورت میں \عددیء{\tfrac{\rho}{\rho}=\tfrac{0}{0}} ہو گا جو غیر معین\فرہنگ{غیر معین}\حاشیہب{undefined}\فرہنگ{undefined} ہے۔یوں  مساوات \حوالہ{مساوات_لاپلاس_ہم_محوری_لاپلاسی_ب_حل} صرف اس صورت مساوات \حوالہ{مساوات_لاپلاس_ہم_محوری_لاپلاسی_ب} کا حل ہو گا اگر \عددیء{\rho \ne 0} ہو۔ان حقائق کو سامنے رکھتے ہوئے لاپلاس مساوات کا حل
\begin{align}
V=V_0 \frac{\ln \frac{b}{\rho} }{\ln \frac{b}{a} } \quad \quad \rho \ne 0
\end{align}
لکھنا زیادہ درست ہو گا۔
\انتہا{مثال} 
%============================

\ابتدا{مثال}\شناخت{مثال_لاپلاس_رداسی_چادروں_کا_کپیسٹر}
اب تصور کرتے ہیں کہ برقی دباو نلکی محدد کے متغیرہ \عددیء{\phi} کے ساتھ تبدیل ہوتا ہے۔اس صورت میں لاپلاس مساوات
\begin{align*}
\frac{1}{\rho^2}\frac{\partial^2 V}{\partial \phi^2}=0
\end{align*}
صورت اختیار کرے گا۔یہاں بھی پہلا حل \عددیء{\rho =\infty} حاصل ہوتا ہے۔ہم یہاں بھی \عددیء{\rho=0} کو جواب کا حصہ تصور نہ کرتے ہوئے مساوات کو \عددیء{\rho^2} سے ضرب دیتے ہوئے اس سے جان چڑاتے ہیں۔یوں
\begin{align*}
\frac{\dif{^2} V}{\dif \phi^2}=0   \quad \quad \rho \ne 0
\end{align*}
رہ جاتا ہے۔دو مرتبہ تکمل لینے سے
\begin{align*}
V=A \phi+B
\end{align*}
حاصل ہوتا ہے۔ایسی دو ہم قوہ سطحیں شکل میں دکھائی گئی ہیں۔آپ دیکھ سکتے ہیں کہ \عددیء{\rho=0} کی صورت میں دونوں چادر آپس میں مل جائیں گی اور ان پر مختلف برقی دباو ممکن نہ ہو گا۔یوں \عددیء{\rho=0} قابل قبول جواب نہیں ہے۔یہاں \عددیء{\phi=0} کو برقی زمین جبکہ \عددیء{\phi=\phi_0} پر \عددیء{V_0} برقی دباو کی صورت میں
\begin{align}
V=\frac{V_0\phi}{\phi_0}  \quad \quad \rho \ne 0
\end{align}
حاصل ہوتا ہے۔اس سے
\begin{align*}
\kvec{E}=-\frac{V_0}{\phi_0 \rho}\aphi
\end{align*}
حاصل ہوتا ہے۔ان چادروں کے کپیسٹنس کا حصول آپ سے حاصل کرنے کو سوال میں کہا گیا ہے۔
\انتہا{مثال}
%========================

\ابتدا{مثال}\شناخت{مثال_لاپلاس_کروی_رداسی_لاپلاسی}
کروی محدد میں \عددیء{\phi} کے ساتھ تبدیلی کو مندرجہ بالا مثال میں دیکھا گیا لہٰذا اسے دوبارہ حل کرنے کی ضرورت نہیں۔ہم پہلے \عددیء{r} اور بعد میں \عددیء{\theta} کے ساتھ تبدیلی کے مسئلوں کو دیکھتے ہیں۔

یہ زیادہ مشکل مسئلہ نہیں ہے لہٰذا آپ ہی سے سوالات کے حصے میں درخواست کی گئی ہے کہ اسے حل کرتے ہوئے برقی دباو کی مساوات 
\begin{align}\label{مساوات_لاپلاس_کروی_رداسی_لاپلاسی_دباو}
V=V_0 \frac{\frac{1}{r}-\frac{1}{b}}{\frac{1}{a}-\frac{1}{b}}
\end{align}
اور کپیسٹنس کی مساوات 
\begin{align}\label{مساوات_لاپلاس_کروی_رداسی_لاپلاسی_کپیسٹنس}
C=\frac{4\pi \epsilon}{\frac{1}{a}-\frac{1}{b}}
\end{align}
حاصل کریں  جہاں \عددیء{r=b} پر برقی زمین اور \عددیء{r=a} پر \عددیء{V_0} برقی دباو ہے اور \عددیء{b > a} ہے۔
\انتہا{مثال}
%=======================

\ابتدا{مثال}
کروی محدد میں \عددیء{\theta} کے ساتھ تبدیل ہوتے برقی دباو کی صورت میں لاپلاس مساوات
\begin{align}
\frac{1}{r^2 \sin \theta} \frac{\dif}{\dif \theta}\left(\sin \theta \frac{\dif V}{\dif \theta}\right)=0
\end{align}
صورت اختیار کرے گی۔اگر \عددیء{r \ne0} اور \عددیء{\sin \theta \ne 0} ہوں تب اس مساوات کو \عددیء{r^2 \sin \theta} سے ضرب دیتے ہوئے
\begin{align}
\frac{\dif}{\dif \theta}\left(\sin \theta \frac{\dif V}{\dif \theta}\right)=0
\end{align}
لکھا جا سکتا ہے۔\عددیء{\sin  \theta} اس صورت صفر کے برابر ہو گا جب \عددیء{\theta=0} یا \عددیء{\theta=\pi} ہوں۔اس کے پہلی بار تکمل سے
\begin{align*}
\sin \theta \frac{\dif V}{\dif \theta}=A
\end{align*}
یا
\begin{align*}
\dif V=\frac{A \dif \theta}{\sin \theta}
\end{align*}
حاصل ہوتا ہے۔دوسری بار تکمل سے
\begin{align}\label{مساوات_لاپلاس_مخروطی_حل}
V=A\int \frac{\dif \theta}{\sin \theta}+B=A \ln \left(\tan \frac{\theta}{2}\right)+B
\end{align}
حاصل ہوتا ہے۔

یہ ہم قوہ سطحیں مخروطی شکل رکھتے ہیں۔اگر \عددیء{\theta=\tfrac{\pi}{2}} پر \عددیء{V=0} اور \عددیء{\theta = \theta_0} پر \عددیء{V=V_0} ہوں جہاں \عددیء{\theta_0 < \tfrac{\pi}{2}} ہے تب ہمیں
\begin{align}\label{مساوات_لاپلاس_مخروطی_حل_ب}
V=V_0 \frac{\ln \left(\tan \frac{\theta}{2} \right)}{\ln \left(\tan \frac{\theta_0}{2} \right)}
\end{align}
حاصل ہوتا ہے۔

آئیں ایسی مخروط اور سیدھی سطح کے مابین کپیسٹنس حاصل کریں جہاں مخروط کی نوک سے انتہائی باریک فاصلے پر سیدھی سطح ہو اور مخروط کا محور اس سطح کے عمود میں ہو۔پہلے برقی شدت حاصل کرتے ہیں۔
\begin{align}
\kvec{E}=-\nabla V=-\frac{1}{r} \frac{\partial V}{\partial \theta} \atheta=-\frac{V_0}{r \sin \theta  \ln \left (\tan \frac{\theta_0}{2} \right)} \atheta
\end{align}
مخروط کی سطح پر سطحی چارج کثافت یوں
\begin{align*}
\rho_S=D_n=-\frac{\epsilon V_0}{r \sin \theta_0  \ln \left (\tan \frac{\theta_0}{2} \right)}
\end{align*}
ہو گا جس سے اس پر چارج
\begin{align*}
Q=-\frac{\epsilon V_0}{\sin \theta_0  \ln \left (\tan \frac{\theta_0}{2} \right)}\int \limits_0^\infty \int \limits_0^{2\pi}\frac{r \sin \theta_0 \dif \phi \dif r}{r}
\end{align*}
ہو گا۔تکمل میں رداس کا حد لامحدود ہونے کی وجہ سے چارج کی قیمت بھی لامحدود حاصل ہوتی ہے جس سے لامحدود کپیسٹنس حاصل ہو گا۔حقیقت میں محدود جسامت کے سطحیں ہی پائی جاتی ہیں لہٰذا ہم رداس کے حدود \عددیء{0} تا \عددیء{r_1} لیتے ہیں۔ایسی صورت میں
\begin{align}\label{مساوات_لاپلاس_مخروطی_کپیسٹنس}
C=\frac{2\pi \epsilon r_1}{\ln \left(\cot \frac{\theta_0}{2} \right)}
\end{align}
حاصل ہوتا ہے۔یاد رہے کہ ہم نے لامحدود سطح سے شروع کیا تھا لہٰذا چارج کی مساوات بھی صرف لامحدود سطح کے لئے درست ہے۔اس طرح مندرجہ بالا مساوات کپیسٹنس کی قریبی قیمت ہو گی نا کہ بالکل درست قیمت۔ 
\انتہا{مثال}
%=======================

\حصہ{پوئسن مساوات کے حل کی مثال}
پوئسن مساوات تب حل کیا جا سکتا ہے جب \عددیء{\rho_h} معلوم ہو۔حقیقت میں عموماً سرحدی برقی دباو وغیرہ معلوم ہوتے ہیں اور ہمیں \عددیء{\rho_h} ہی درکار ہوتی ہے۔ہم پوئسن مساوات حل کرنے کی خاطر ایسی مثال لیتے ہیں جہاں ہمیں \عددیء{\rho_h} معلوم ہو۔

\اصطلاح{سلیکان}\فرہنگ{سلیکان}\حاشیہب{silicon}\فرہنگ{silicon} کی پتری میں \عددیء{p} اور \عددیء{n} اقسام کے مواد کی ملاوٹ سے \عددیء{p} اور \عددیء{n} سلیکان پیدا کیا جاتا ہے۔ایک ہی سلیکان پتری پر آپس میں جڑے ہوئے \عددیء{p} اور \عددیء{n} خطے \اصطلاح{ڈایوڈ}\فرہنگ{ڈایوڈ}\حاشیہب{diode}\فرہنگ{diode} کو جنم دیتے ہیں۔\عددیء{x} محدد پر رکھے ایسے ہی ایک ڈایوڈ کی بات کرتے ہوئے آگے بڑھتے ہیں۔تصور کریں کہ \عددیء{x<0} خطہ \عددیء{p} اور \عددیء{x>0} خطہ \عددیء{n} قسم کا ہے۔مزید یہ کہ دونوں جانب ملاوٹ کی مقدار یکساں ہے۔آپ کو یاد ہو گا کہ \عددیء{p} یا \عددیء{n} خطہ ازخود غیر چارج شدہ ہوتا ہے البتہ \عددیء{p} خطے میں \اصطلاح{آزاد خول}\فرہنگ{خول!آزاد}\حاشیہب{free holes}\فرہنگ{holes!free} اور \عددیء{n} خطے میں \اصطلاح{آزاد الیکٹران}\فرہنگ{آزاد الیکٹران}\حاشیہب{free electrons}\فرہنگ{electrons!free} پائے جاتے ہیں۔ آزاد خول اور آزاد الیکٹران حرکت کر سکتے ہیں۔جس لمحہ یہ آپس میں جڑے خطے وجود میں آتے ہیں، اس وقت آزاد خول صرف \عددیء{p} جانب جبکہ آزاد الیکٹران صرف \عددیء{n} جانب پائے جاتے ہیں۔یوں اس لمحے ہی  آزاد خول \عددیء{p} سے \عددیء{n} جانب اور آزاد الیکٹران \عددیء{n} سے \عددیء{p} جانب  \اصطلاح{نفوذ}\فرہنگ{نفوذ}\حاشیہب{diffusion}\فرہنگ{diffusion} کے ذریعے حرکت کرنا شروع کر دیتے ہیں۔چارج کے اس حرکت سے جلد \عددیء{p} اور \عددیء{n} کے سرحد کے دونوں جانب الٹ قطب کا چارج جمع ہونے شروع ہو جاتا ہے۔یوں دو چادر کپیسٹر پر چارج کی طرح، سرحد کے دائیں یعنی \عددیء{x>0} جانب مثبت جبکہ اس کے بائیں جانب منفی چارج جمع ہو جاتا ہے۔یہ چارج کپیسٹر کے چادروں کے درمیان برقی میدان کی طرح \عددیء{\kvec{E}=-E\ax} پیدا کرتا ہے جو بائیں سے دائیں جانب آزاد خول کے حرکت اور دائیں سے بائیں جانب آزاد الیکٹران کے حرکت  کو روکتا ہے۔جب تک برقی میدان \عددیء{\kvec{E}} چارج کے اس حرکت کو نہ روک سکے اس وقت تک چارج کا نفوذ جاری رہے گا جس سے سرحد کے دونوں جانب چارج کا انبار بڑھتا رہے گا جس سے \عددیء{\kvec{E}} بڑھتی رہے گی۔آخر کار  \عددیء{\kvec{E}} کی قیمت اتنی ہو جائے گی کہ یہ نفوذ کو مکمل طور پر روک دے گا۔آئیں برقی سکون کے اس حال پر غور کریں۔ابتدا میں \عددیء{p} اور \عددیء{n} خطے دونوں غیر چارج شدہ تھے البتہ برقی سکون کی حالت اختیار کرنے کے بعد صاف ظاہر ہے کہ سرحد کے دائیں جانب مثبت جبکہ اس کے بائیں جانب منفی چارج پایا جاتا ہے۔سرحد کے  دائیں جانب مثبت چارج دو وجوہات کی بنا ہے۔کچھ تو آزاد خول اس جانب منتقل ہوئے ہیں اور کچھ یہاں سے آزاد الیکٹران کی نفوذ سے مثبت ایٹم یہاں رہ گئے ہیں۔اسی طرح سرحد کے دوسری جانب منفی چارج کچھ تو آزاد الیکٹران کی آمد اور کچھ یہاں سے آزاد خول کے اخراج سے  منفی ایٹموں کے رہ جانے کی وجہ سے ہے۔سرحد کے دونوں جانب الٹ قطب کے چارج میں قوت کشش پائی جاتی ہے جو انہیں سرحد کے قریب ہی رکھتے ہیں۔

سرحد کے دونوں جانب چارج کے انبار کو شکل میں دکھایا گیا ہے۔اس طرح کے انبار کو کئی مساوات سے ظاہر کرنا ممکن ہے جن میں غالباً سب سے سادہ مساوات
\begin{align}\label{مساوات_لاپلاس_ڈایوڈ_چارج-کثافت}
\rho=2 \rho_0 \sech \frac{x}{a} \tanh \frac{x}{a}
\end{align}
ہے جہاں زیادہ سے زیادہ چارج کثافت \عددیء{\rho_0} ہے جو \عددیء{x=0.881a} پر پائی جاتی ہے۔آئیں اس چارج کثافت کے لئے پوئسن مساوات
\begin{align*}
\nabla^2 V=-\frac{2 \rho_0}{\epsilon} \sech \frac{x}{a} \tanh \frac{x}{a}
\end{align*}
یعنی
\begin{align*}
\frac{\dif^{\hspace{1pt}2} V}{\dif x^2}=-\frac{2 \rho_0}{\epsilon} \sech \frac{x}{a} \tanh \frac{x}{a}
\end{align*}
 حل کریں۔پہلی بار تکمل لیتے ہوئے
\begin{align*}
\frac{\dif V}{\dif x}=\frac{2\rho_0 a}{\epsilon} \sech \frac{x}{a}+A
\end{align*}
حاصل ہوتا  ہے جسے
\begin{align*}
E_x=-\frac{\dif V}{\dif x}=\frac{2\rho_0 a}{\epsilon} \sech \frac{x}{a}-A
\end{align*}
بھی لکھا جا سکتا ہے۔تکمل کے مستقل \عددیء{A} کی قیمت اس حقیقت سے حاصل کی جا سکتی ہے کہ سرحد سے دور کسی قسم کا چارج کثافت یا برقی میدان نہیں پایا جاتا لہٰذا \عددیء{x \to \mp \infty} پر \عددیء{ٰE_x \to 0} ہو گا جس سے \عددیء{A=0} حاصل ہوتا ہے لہٰذا
 \begin{align}\label{مساوات_لاپلاس_ڈایوڈ_برقی_میدان}
E_x=-\frac{\dif V}{\dif x}=-\frac{2\rho_0 a}{\epsilon} \sech \frac{x}{a}
\end{align}
کے برابر ہے۔دوسری بار تکمل لیتے ہوئے
\begin{align*}
V=\frac{4 \rho_0 a^2}{\epsilon} \tan^{-1} e^{\frac{x}{a}}+B
\end{align*}
حاصل ہوتا ہے۔ہم برقی زمین کو عین سرحد پر لیتے ہیں۔ایسا کرنے سے \عددیء{B=-\frac{\rho_0a^2\pi}{\epsilon}} حاصل ہوتا ہے۔یوں
\begin{align}\label{مساوات_لاپلاس_ڈایوڈ_برقی_دباو}
V=\frac{4 \rho_0 a^2}{\epsilon} \left(\tan^{-1} e^{\frac{x}{a}}-\frac{\pi}{4}\right)
\end{align}
کے برابر ہو گا۔

شکل میں مساوات \حوالہ{مساوات_لاپلاس_ڈایوڈ_چارج-کثافت}، مساوات \حوالہ{مساوات_لاپلاس_ڈایوڈ_برقی_میدان} اور مساوات \حوالہ{مساوات_لاپلاس_ڈایوڈ_برقی_دباو} دکھائے گئے ہیں جو بالترتیب حجمی چارج کثافت، برقی میدان کی شدت اور برقی دباو دیتے ہیں۔

سرحد کے دونوں جانب کے مابین برقی دباو \عددیء{V_0} کو مساوات \حوالہ{مساوات_لاپلاس_ڈایوڈ_برقی_دباو} کی مدد سے یوں حاصل کیا جا سکتا ہے۔
\begin{align}\label{مساوات_لاپلاس_ڈایوڈ_اندرونی_دباو}
V_0=V_{x \to +\infty}-V_{x \to -\infty}=\frac{2\pi\rho_0 a^2}{\epsilon}
\end{align}
سرحد کے ایک جانب کل چارج کو مساوات \حوالہ{مساوات_لاپلاس_ڈایوڈ_چارج-کثافت} کی مدد سے حاصل کیا جا سکتا ہے۔یوں کل مثبت چارج
\begin{align}\label{مساوات_لاپلاس_ڈایوڈ_کل_چارج_الف}
Q=S \int_0^{\infty} 2 \rho_0 \sech \frac{x}{a}\tanh \frac{x}{a} \dif x=2\rho_0 a S
\end{align}
حاصل ہوتا ہے جہاں ڈایوڈ کا \اصطلاح{رقبہ عمودی تراش}\فرہنگ{رقبہ!عمودی تراش}\حاشیہب{cross sectional area}\فرہنگ{area!cross sectional} \عددیء{S} ہے۔مساوات \حوالہ{مساوات_لاپلاس_ڈایوڈ_اندرونی_دباو} سے \عددیء{a} کی قیمت مساوات \حوالہ{مساوات_لاپلاس_ڈایوڈ_کل_چارج_الف} میں پر کرنے سے
\begin{align}\label{مساوات_لاپلاس_ڈایوڈ_کل_چارج_ب}
Q=S \sqrt{\frac{2\rho_0 \epsilon V_0}{\pi}}
\end{align} 
لکھا جا سکتا ہے۔اس مساوات سے کپیسٹنس کی قیمت \عددیء{C=\tfrac{Q}{V_0}} لکھ کر نہیں حاصل کی جا سکتی البتہ
\begin{align*}
I&=\frac{\dif Q}{\dif t} =C \frac{\dif V_0}{\dif t}
\end{align*}
سے
\begin{align*}
C=\frac{\dif Q}{\dif V_0}
\end{align*}
لکھا جا سکتا ہے  لہٰذا مساوات \حوالہ{مساوات_لاپلاس_ڈایوڈ_کل_چارج_ب} کا تفرق لیتے ہوئے
\begin{align*}
C=\sqrt{\frac{\rho_0 \epsilon}{2\pi V_0}} S=\frac{\epsilon S}{2\pi a}
\end{align*}
حاصل ہوتا ہے۔اس مساوات  کے پہلے جزو سے ظاہر ہے کہ برقی دباو بڑھانے سے کپیسٹنس کم ہو گی۔اسے اس طرح سمجھا جا سکتا ہے کہ زیادہ برقی دباو کی صورت میں سرحد کے دونوں جانب چارجوں کا آپس میں فاصلہ بڑھ جاتا ہے۔یہ اسی طرح ہے جیسے دو چادر کپیسٹر کے چادروں کا آپس میں فاصلہ بڑھا دیا جائے جس سے کپیسٹنس کم ہی ہوتی ہے۔مساوات کے دوسرے جزو سے یہ اخذ کیا جا سکتا ہے کہ ڈایوڈ بالکل ایسے  دو چادر کپیسٹر کی طرح ہے جس کے چادر کا رقبہ \عددیء{S} اور چادروں کے مابین فاصلہ \عددیء{2\pi a}  ہو۔ 
