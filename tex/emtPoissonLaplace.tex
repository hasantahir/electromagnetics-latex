\باب{پوئسن اور لاپلاس مساوات}
گاوس کے قانان کی نقطہ شکل
\begin{align}
\nabla \cdot \kvec{D}=\rho_h
\end{align}
میں \عددیء{\kvec{D}=\epsilon \kvec{E}} اور حاصل جواب میں \عددیء{\kvec{E}=-\nabla V} پر کرنے سے
\begin{align*}
\nabla \cdot (\epsilon \kvec{E})=-\nabla \cdot (\epsilon \nabla V)=\rho_h 
\end{align*}
یعنی
\begin{align}\label{مساوات_پوئسن_نقطہ}
\nabla \cdot \nabla V=-\frac{\rho_h}{\epsilon}
\end{align}
حاصل ہوتا ہے  جہاں ہر طرف یکساں\فرہنگ{یکساں!ہر طرف}\حاشیہب{homogeneous}\فرہنگ{homogeneous} خطے میں \عددیء{\epsilon} اٹل مقدار ہے۔مساوات \حوالہ{مساوات_پوئسن_نقطہ} \اصطلاح{پوئسن}\فرہنگ{پوئسن مساوات}\حاشیہب{Poisson equation}\فرہنگ{Poisson equation} مساوات  کہلاتا ہے۔

آئیں کارتیسی محدد میں پوئسن مساوات کی شکل حاصل کریں۔یاد رہے کہ کسی بھی متغیرہ \عددیء{\kvec{A}=A_x\ax+A_y\ay+A_z\az} کے لئے
\begin{align*}
\nabla \cdot \kvec{A}=\frac{\partial A_x}{\partial x}+\frac{\partial A_y}{\partial y}+\frac{\partial A_z}{\partial z}
\end{align*}  
کے برابر ہوتا ہے۔اب چونکہ
\begin{align*}
\nabla{V}=\frac{\partial V}{\partial x}\ax+\frac{\partial V}{\partial y}\ay+\frac{\partial V}{\partial z}\az
\end{align*}
 کے برابر ہے لہٰذا
\begin{gather}
\begin{aligned}
\nabla \cdot \nabla V&=\frac{\partial }{\partial x}\left(\frac{\partial V}{\partial x}\right)+\frac{\partial }{\partial y}\left(\frac{\partial V}{\partial y}\right)+\frac{\partial }{\partial z}\left(\frac{\partial V}{\partial z}\right)\\
&=\frac{\partial^2 V}{\partial x^2}+\frac{\partial^2 V}{\partial y^2}+\frac{\partial^2 V}{\partial z^2}
\end{aligned}
\end{gather}
 ہو گا۔

عموماً \عددیء{\nabla \cdot \nabla} کو \عددیء{\nabla^2} لکھا جاتا ہے۔اس طرح پوئسن مساوات کی کارتیسی شکل
\begin{align}\label{مساوات_لاپلاس_پوئسن_کارتیسی_شکل}
\nabla^2 V=\frac{\partial^2 V}{\partial x^2}+\frac{\partial^2 V}{\partial y^2}+\frac{\partial^2 V}{\partial z^2}=-\frac{\rho_h}{\epsilon}
\end{align}
حاصل ہوتی ہے۔ 

حجمی چارج کثافت کی غیر موجودگی، یعنی \عددیء{\rho_h =0} کی صورت میں مساوات \حوالہ{مساوات_پوئسن_نقطہ}
\begin{align}\label{مساوات_لاپلاس_لاپلاس_نقطہ_شکل}
\nabla^2 V=0
\end{align}
صورت اختیار کر لے گی جسے \اصطلاح{لاپلاس}\فرہنگ{لاپلاس مساوات}\حاشیہب{Laplace equation}\فرہنگ{Laplace equation} مساوات کہتے ہیں۔جس حجم کے لئے لاپلاس کی مساوات لکھی گئی ہو اس حجم میں حجمی چارج کثافت صفر ہوتا ہے البتہ اس حجم کی سرحد پر نقطہ چارج یا سطحی چارج کثافت پائی جا سکتیں ہیں۔عموماً سطح پر موجود چارج سے حجم میں پیدا میدان ہی حاصل کرنا مطلوب ہوتا ہے۔کارتیسی محدد میں لاپلاس کی مساوات
\begin{align}\label{مساوات_لاپلاس_لاپلاس_کارتیسی_شکل}
\nabla^2 V=\frac{\partial^2 V}{\partial x^2}+\frac{\partial^2 V}{\partial y^2}+\frac{\partial^2 V}{\partial z^2}=0
\end{align}
صورت رکھتی ہے۔\عددیء{\nabla^2} کو لاپلاسی عامل\فرہنگ{لاپلاسی عامل}\حاشیہب{Laplacian operator}\فرہنگ{Laplacian operator} کہا جاتا ہے۔

لاپلاس مساوات کہتا ہے کہ کسی بھی چارج سے خالی حجم میں ہر صورت \عددیء{\nabla^2=0} ہو گا۔حجم کی شکل کچھ بھی ہو سکتی ہے اور اس کے سرحد پر کسی بھی قسم کا چارج ہو سکتا ہے۔یہ ایک دلچسپ حقیقت ہے۔حجم کے سرحد پر عموماً ایک یا ایک سے زیادہ موصل سطحیں ہوتی ہیں جن پر برقی دباو \عددیء{V_0}، \عددیء{V_1}، \عددیء{V_2} وغیرہ پایا جاتا ہے اور حجم کے اندر میدان کا حصول درکار ہوتا ہے۔کبھی کبھار موصل سطح پر چارج یا \عددیء{\kvec{E}} معلوم ہو گا جس سے حجم کے اندر میدان درکار ہو گا۔اسی طرح کبھی کبھار سرحد پر ایک جگہ چارج اور اس پر دوسری جگہ برقی دباو اور اس پر تیسرے  جگہ عمودی بہاو دیا گیا ہو گا جبکہ حجم کے اندر کے متغیرات درکار ہوں گے۔اس کے برعکس ایسا بھی ممکن ہے کہ حجم میں میدان یا برقی دباو معلوم ہو اور ان معلومات سے سرحد پر چارج یا بہاو یا برقی دباو حاصل کرنا ضروری ہو گا۔

اس باب میں ہم ایسی کئی مثالیں دیکھیں گے لیکن پہلے یہ حقیقت جاننا ضروری ہے کہ مساوات \حوالہ{مساوات_لاپلاس_لاپلاس_کارتیسی_شکل} کا کوئی بھی جواب ان تمام اقسام کے سرحدی معلومات کے لئے درست ہو گا۔یہ انتہائی تشویشناک بات ہو گی اگر دو مختلف طریقوں سے لاپلاس مساوات کے جوابات حاصل کرنے کے بعد معلوم ہو کہ ان میں سے ایک ٹھیک اور دوسرا غلط جواب ہے۔آئیں اس حقیقت کا ثبوت دیکھیں کہ کسی بھی سرحدی حقائق کو مد نظر رکھتے ہوئے لاپلاس مساوات کا صرف اور صرف ایک ہی جواب حاصل ہوتا ہے۔

\حصہ{مسئلہ یکتائی}
تصور کریں کہ ہم دو مختلف طریقوں سے لاپلاس مساوات کے دو جوابات \عددیء{V_1} اور \عددیء{V_2} حاصل کرتے ہیں۔یہ دونوں جوابات لاپلاس مساوات پر پورا اترتے ہیں لہٰذا
\begin{align*}
\nabla^2 V_1&=0\\
\nabla^2 V_2&=0
\end{align*} 
لکھا جا سکتا ہے جس سے
\begin{align}\label{مساوات_لاپلاس_دو_جوابات_الف}
\nabla^2 (V_1-V_2)=0
\end{align}
حاصل ہوتا ہے۔اب اگر سرحد پر برقی دباو \عددیء{V_s} ہو تب دونوں جوابات سرحد پر یہی جواب دیں گے یعنی سرحد پر
\begin{align*}
V_{1s}=V_{2s}=V_s
\end{align*}
یا
\begin{align*}
V_{1s}-V_{2s}=0
\end{align*}
ہو گا۔صفحہ \حوالہصفحہ{مساوات_توانائی_توانائی_ضرب_برقی_بہاو_کی_ڈھلان} پر مساوات \حوالہ{مساوات_توانائی_توانائی_ضرب_برقی_بہاو_کی_ڈھلان}
\begin{align*}
\nabla \cdot  (V \kvec{D})=V (\nabla \cdot \kvec{D})+\kvec{D} \cdot (\nabla V)
\end{align*}
کا ذکر کیا گیا جو کسی بھی مقداری \عددیء{V} اور کسی بھی سمتیہ \عددیء{\kvec{D}} کے لئے درست ہے۔موجودہ استعمال کے لئے ہم \عددیء{V_1-V_2} کو مقداری اور \عددیء{\nabla(V_1-V_2)} کو سمتیہ لیتے ہوئے
\begin{align*}
\nabla \cdot  [(V_1-V_2) \nabla(V_1-V_2)]&=(V_1-V_2) [\nabla \cdot \nabla(V_1-V_2)]+\nabla(V_1-V_2) \cdot \nabla (V_1-V_2)\\
&=(V_1-V_2) [\nabla^2(V_1-V_2)]+[\nabla(V_1-V_2)]^2
\end{align*}
حاصل ہوتا ہے جس کا تکمل پورے حجم کے لئے
\begin{align}\label{مساوات_لاپلاس_یکتائی_ثبوت_الف}
\int\limits_{\textrm{حجم}} \nabla \cdot  [(V_1-V_2) \nabla(V_1-V_2)] \dif h=\int\limits_{\textrm{حجم}}(V_1-V_2) [\nabla^2(V_1-V_2)] \dif h+\int\limits_{\textrm{حجم}}[\nabla(V_1-V_2)]^2 \dif h
\end{align}
ہو گا۔صفحہ \حوالہصفحہ{مساوات_گاوس_مسئلہ_پھیلاو_تکمل_شکل} پر مساوات \حوالہ{مساوات_گاوس_مسئلہ_پھیلاو_تکمل_شکل} مسئلہ پھیلاو بیان کرتا ہے جس کے مطابق کسی بھی حجمی تکمل کو  بند سطحی تکمل میں تبدیل کیا جا سکتا ہے جہاں حجم کی سطح پر سطحی تکمل حاصل کیا جاتا ہے۔یوں مندرجہ بالا مساوات کے بائیں ہاتھ کو سطحی تکمل میں تبدیل کرتے ہوئے
\begin{align*}
\int\limits_{\textrm{حجم}} \nabla \cdot  [(V_1-V_2) \nabla(V_1-V_2)] \dif h=\oint\limits_{\textrm{سطح}} [(V_{1s}-V_{2s}) \nabla(V_{1s}-V_{2s})] \cdot \dif \kvec{S}=0
\end{align*}
حاصل ہوتا ہے جہاں سرحدی سطح پر \عددیء{V_{1s}=V_{2s}} ہونے کی بنا پر \عددیء{V_{1s}-V_{2s}=0} ہے اور صفر کا تکمل صفر ہی ہوتا ہے۔مساوات \حوالہ{مساوات_لاپلاس_یکتائی_ثبوت_الف} میں دائیں ہاتھ پہلے جزو میں مساوات \حوالہ{مساوات_لاپلاس_دو_جوابات_الف} کے تحت \عددیء{\nabla^2(V_1-V_2)=0} ہے اور صفر کا تکمل صفر ہی ہوتا ہے۔اس طرح مساوات \حوالہ{مساوات_لاپلاس_یکتائی_ثبوت_الف} سے
\begin{align*}
\int\limits_{\textrm{حجم}}[\nabla(V_1-V_2)]^2 \dif h=0
\end{align*}
حاصل ہوتا ہے۔

کسی بھی تکمل کا جواب صرف دو صورتوں میں صفر کے برابر ہو سکتا ہے۔پہلی صورت یہ ہے کہ کچھ خطے میں تکمل کی قیمت مثبت اور کچھ خطے میں اس کی قیمت منفی ہو۔اگر مثبت اور منفی حصے بالکل برابر ہوں تب تکمل صفر کے برابر ہو گا۔موجودہ صورت میں \عددیء{[\nabla(V_1-V_2)]^2} کا تکمل لیا جا رہے ہے اور کسی بھی متغیر کا مربع کسی صورت منفی نہیں ہو سکتا لہٰذا موجودہ تکمل میں ایسا ممکن نہیں ہے۔تکمل صفر ہونے کی دوسری صورت یہ ہے کہ صفر کا تکمل حاصل کیا جا رہا ہو لہٰذا
\begin{align*}
[\nabla(V_1-V_2)]^2 =0
\end{align*}
ہی ہو گا یعنی
\begin{align*}
\nabla (V_1-V_2)=0
\end{align*}
کے برابر ہے۔

اب \عددیء{\nabla (V_1-V_2)=0} کا مطلب ہے کہ \عددیء{V_1-V_2} کی ڈھلان ہر صورت صفر کے برابر ہے۔یہ تب ہی ممکن ہے جب \عددیء{V_1-V_2} کی قیمت کسی بھی محدد کے ساتھ تبدیل نہ ہو یعنی اگر تکمل کے پورے خطے میں
\begin{align*}
V_1-V_2=\textrm{اٹل قیمت}
\end{align*}
ہو۔حجم کے سرحد پر بھی یہ درست ہو گا۔مگر سرحد پر
\begin{align*}
V_1-V_2=V_{1s}-V_{2s}=0
\end{align*}
کے برابر ہے لہٰذا یہ اٹل قیمت ازخود صفر ہے۔یوں
\begin{align}
V_1=V_2
\end{align}
ہو گا۔اس کا مطلب  ہے کہ دونوں جوابات بالکل برابر ہیں۔

مسئلہ یکتائی کو پوئسن مساوات کے لئے بھی بالکل اسی طرح ثابت کیا جا سکتا ہے۔پوئسن مساوات کے دو جوابات \عددیء{V_1} اور \عددیء{V_2} پوئسن مساوات پر پورا اتریں گے لہٰذا \عددیء{\nabla^2 V_1=-\tfrac{\rho_h}{\epsilon}} اور \عددیء{\nabla^2 V_2=-\tfrac{\rho_h}{\epsilon}} لکھے جا سکتے ہیں جن سے \عددیء{\nabla^2(V_1-V_2)=0} حاصل ہوتا ہے۔سرحد پر اب بھی \عددیء{{V_{1s}-V_{2s}= 0}} ہو گا۔یہاں سے آگے ثبوت بالکل یکتائی لاپلاس کی ثبوت کی طرح ہے۔

مسئلہ یکتائی کے تحت سرحدی حقائق کے لئے حاصل کئے  گئے پوئسن یا لاپلاس مساوات کے جوابات ہو صورت برابر ہوں گے۔یہ ممکن نہیں کہ دو مختلف جوابات حاصل کئے جائیں۔ 

\حصہ{نلکی اور کروی محدد میں لاپلاس کی مساوات} 
نلکی محدد میں ڈھلان کی مساوات صفحہ \حوالہصفحہ{مساوات_توانائی_ڈھلان_نلکی} پر مساوات \حوالہ{مساوات_توانائی_ڈھلان_نلکی} دیتا ہے جس سے 
\begin{gather}
\begin{aligned}\label{مساوات_لاپلاس_نلکی_پھیلاو_الف}
\nabla V &= \frac{\partial V}{\partial \rho} \arho+\frac{1}{\rho}\frac{\partial V}{\partial \phi}  \aphi+\frac{\partial V}{\partial z}\az\\
&=-E_\rho \arho-E_\phi \aphi-E_z \az 
\end{aligned}
\end{gather}
لکھتے  ہیں جہاں \عددیء{\kvec{E}=-\nabla V} کا استعمال کیا گیا۔نلکی محدد میں پھیلاو کی مساوات صفحہ \حوالہصفحہ{مساوات_گاوس_نلکی_عمومی_پھیلاو} پر مساوات \حوالہ{مساوات_گاوس_نلکی_عمومی_پھیلاو} دیتا ہے۔اسی مساوات کو سمتیہ \عددیء{\kvec{E}} کے لئے
 \begin{align*}
\nabla \cdot \kvec{E}=\frac{1}{\rho}\frac{\partial (\rho E_{\rho})}{\partial \rho}+\frac{1}{\rho}\frac{\partial E_{\phi}}{\partial \phi}  +  \frac{\partial E_{z}}{\partial z}
\end{align*}
لکھتے ہیں۔اس میں بائیں ہاتھ \عددیء{\kvec{E}=-\nabla V}  اور دائیں ہاتھ مساوات \حوالہ{مساوات_لاپلاس_نلکی_پھیلاو_الف} سے قیمتیں پر کرتے ہوئے
\begin{align*}
\nabla \cdot \nabla V=\frac{1}{\rho}\frac{\partial }{\partial \rho}\left(\rho \frac{\partial V}{\partial \rho}\right)
+\frac{1}{\rho}\frac{\partial }{\partial \phi}\left(\frac{1}{\rho}\frac{\partial V}{\partial \phi}  \right) 
+  \frac{\partial}{\partial z} \left(\frac{\partial V}{\partial z} \right)
\end{align*}
حاصل ہوتا ہے جہاں دونوں جانب منفی علامت کٹ جاتے ہیں۔اس کو یوں
\begin{align}
\nabla^2 V=\frac{1}{\rho}\frac{\partial }{\partial \rho}\left(\rho \frac{\partial V}{\partial \rho}\right)
+\frac{1}{\rho^2}\left(\frac{\partial^2 V}{\partial \phi^2}  \right) 
+  \frac{\partial^2 V}{\partial z^2}\quad {\textrm{نلکی}}
\end{align}
 لکھا جا سکتا ہے جو نلکی محدد میں لاپلاسی مساوات ہے۔

کروی محدد میں بالکل اسی
\begin{align}\label{مساوات_لاپلاس_کروی_لاپلاسی}
\nabla^2 V=\frac{1}{r^2} \frac{\partial}{\partial r} \left(r^2\frac{\partial V}{\partial r} \right)+\frac{1}{r^2 \sin \theta} \frac{\partial}{\partial \theta} \left(\sin \theta \frac{\partial V}{\partial \theta}  \right)+\frac{1}{r^2 \sin \theta}\frac{\partial^2 V}{\partial \phi^2} \quad {\textrm{کروی}}
\end{align}
جبکہ عمومی محدد میں
\begin{align}\label{مساوات_لاپلاس_عمومی_لاپلاسی}
\nabla^2 V=\frac{1}{k_1 k_2 k_3}\left[\frac{\partial}{\partial u}\left(\frac{k_2 k_3}{k_1}\frac{\partial V}{\partial u} \right)+\frac{\partial}{\partial v}\left(\frac{k_1 k_3}{k_2}\frac{\partial V}{\partial v} \right) +\frac{\partial}{\partial w}\left(\frac{k_1 k_2}{k_3}\frac{\partial V}{\partial w} \right)\right] \quad{\textrm{عمومی}}
\end{align}
حاصل کی جا سکتی ہے۔
%==================

\ابتدا{مشق}
مساوات \حوالہ{مساوات_لاپلاس_کروی_لاپلاسی} حاصل کریں۔
\انتہا{مشق}
%============================

