\باب{سوالات}
\section*{سمتیات}
\ابتدا{سوال}
سمتیہ \عددی{\kvec{A}=-2\ax+1\ay+7\az} اور \عددی{\kvec{B}=3\ax+5\ay-2\az} ہیں۔مندرجہ ذیل حاصل کریں: (الف) \عددی{2\kvec{A}-3\kvec{B}} اور اسی کی سمت میں اکائی سمتیہ؛ (ب) \عددی{2 \kvec{A}-5\kvec{B}+3\ax}؛ (پ) \عددی{\abs{3\kvec{A}}\abs{2\kvec{B}}(\kvec{B}-\kvec{A})}

جوابات: \عددی{-13\ax-13\ay+8\az}، \عددی{-0.648\ax-0.648\ay-0.399\az}، \عددی{28.3}، \عددی{1359\ax+1087\ay+1359\az}
\انتہا{سوال}

%=============================
\ابتدا{سوال}
نقطہ \عددی{A(1,-2,3)}، \عددی{B(3,-1,2)} اور \عددی{C(7,5,-4)} دیے گئے ہیں۔(الف) محدد کے مرکز سے \عددی{A} تک سمتیہ لکھیں؛ (ب) مرکز سے لکیر \عددی{AB} کے وسط تک سمتیہ لکھیں؛ (پ) اسی سمت میں اکائی سمتیہ لکھیں؛ (ت) تکون \عددی{ABC} کا احاطہ دریافت کریں۔

جوابات: \عددی{\ax-2\ay+3\az}، \عددی{2\ax-1.5\ay+2.5\az}، \عددی{0.566\ax-0.424\ay-0.707\az}، \عددی{23.4} 
\انتہا{سوال}
%==========================
\ابتدا{سوال}
مرکز سے نقطہ \عددیء{A} تک سمتیہ \عددیء{2\ax+\ay+3\az} ہے جبکہ مرکز سے \عددیء{\tfrac{2}{3}\ax-\tfrac{2}{3}\ay+\tfrac{1}{3}\az} اکائی سمتیہ کی سمت میں نقطہ \عددیء{B} پایا جاتا ہے۔دونوں نقطوں کے درمیان \عددیء{4} فاصلہ  ہونے کی صورت میں نقطہ \عددیء{B} دریافت کریں۔

جوابات:\عددیء{(2.57,-2.57,1.28)}
\انتہا{سوال}
%===============================
\ابتدا{سوال}
سمتی میدان \عددیء{\kvec{M}=(x+y^2)\ax+2(xy+3)\ay+4z^2\az} دیا گیا ہے۔نقطہ \عددیء{A(2,-3,1)} پر اس میدان کی قیمت حاصل کریں۔اسی نقطے پر میدان کی سمت میں اکائی سمتیہ دریافت کریں۔ایسی سطح جس پر \عددی{\abs{\kvec{M}}=5} ہو کی مساوات حاصل کریں۔اس سطح پر \عددیء{y=2} اور \عددیء{z=-1} ہونے کی صورت میں حاصل لکیر کی مساوات حاصل کریں۔

جوابات:\عددی{\kvec{M}=11\ax-6\ay+4\az}، \عددی{(0.836\ax-0.456\ay+0.304\az)}، \\ \عددی{x^2+y^2+2xy^2+4x^2y^2+24xy+16z^4-11=0}، \عددی{17x^2+56x+9=0}
\انتہا{سوال}
%===============================
\ابتدا{سوال}
سمتی میدان \عددی{\kvec{B}=2x^2\ax-3y(x+2z)\ay+5\az} اور \عددیء{\kvec{M}=(x+y+z)\ax+\tfrac{y}{x}\ay+xy\az} دیے گئے ہیں۔نقطہ \عددی{N(2,-3,-1)} پر \عددی{\kvec{B}} اور \عددیء{\kvec{M}} حاصل کریں۔اسی نقطے پر سمتیہ \عددیء{2\kvec{B}-\kvec{M}} کی سمت میں اکائی سمتیہ حاصل کریں۔

جوابات:\عددی{\kvec{B}=8\ax+5\az}، \عددی{\kvec{M}=-2\ax-1.5\ay-2\az}، \عددی{0.830\ax+0.069\ay+0.553\az}
\انتہا{سوال}
%================
\ابتدا{سوال}
 نقطہ \عددیء{N(2,-3,7))} پر میدان \عددیء{\kvec{M}=\tfrac{16}{x^2+y^2}(x\ax+y\ay)} کی سمت میں اکائی سمتیہ \عددی{\kvec{a}_M} دریافت کریں۔نقطہ \عددیء{N} پر \عددیء{\ax} اور \عددیء{\kvec{M}} کے درمیان زاویہ حاصل کریں۔اسی طرح نقطہ \عددیء{N} پر \عددیء{\ay} اور \عددیء{\kvec{M}} کے درمیان زاویہ حاصل کریں۔

جوابات:\عددی{\kvec{a}_M=0.555\ax-0.832\ay}، \عددی{56.3^{\circ}}، \عددی{33.7^{\circ}}
\انتہا{سوال}
%========================
\ابتدا{سوال}
میدان \عددیء{\kvec{M}=\tfrac{16}{x^2+y^2}(x\ax+y\ay)} کا مندرجہ ذیل دو درجی تکمل \عددی{y=3} سطح پر حاصل کریں۔
\begin{align*}
\int_{0}^{3} \int_{0}^{2} \kvec{M} \dif x \dif z\cdot \ax 
\end{align*}

جواب: \عددیء{24 \ln \frac{13}{9}}
\انتہا{سوال}
%========================
\ابتدا{سوال}
غیر سمتی ضرب استعمال کرتے ہوئے تکون \عددیء{ABC} میں زاویہ \عددی{A} اور \عددی{C} حاصل کریں۔تکون کے کونے \عددی{A(3,1,2)}، \عددی{B(4,6,2)} اور
 \عددی{C(1,4,-2)} ہیں۔

جوابات:\عددی{61.74^{\circ}} ، \عددیء{56.51^{\circ}}
\انتہا{سوال}
%===================
\ابتدا{سوال}
نقطے \عددیء{A(4,1,2)}، \عددیء{B(-2,4,3)} اور \عددیء{C(2,3,-1)} دیے گئے ہیں۔سمتیہ \عددی{\kvec{R}_{BA}} اور \عددی{\kvec{R}_{CA}} حاصل کریں۔دوسری سمتیہ  پر  پہلی سمتیہ کے \اصطلاح{عمودی سائے}\فرہنگ{عمودی سائے}\فرہنگ{سایہ!عمودی}\حاشیہب{projection}\فرہنگ{projection} کی لمبائی دریافت کریں۔لکیر \عددیء{AB} کے درمیانے نقطے سے لکیر \عددی{AC} کے درمیانے نقطے تک سیدھا سمتیہ حاصل کریں۔

جوابات: \عددی{-6\ax+3\ay+\az}، \عددی{-2\ax+2\ay-3\az}، \عددی{4.12}، \عددی{2\ax-0.5\ay-2\az}
\انتہا{سوال}
%==================
\ابتدا{سوال}
سمتیہ \عددیء{\kvec{M}=5\ax-3\ay+2\az} کا وہ حصہ حاصل کریں جو سمتیہ \عددی{\kvec{P}=-7\ax+2\ay-6\az} کے متوازی ہے۔وہ حصہ حاصل کریں جو اس کے عمودی ہے۔ 

جوابات: \عددی{4.17\ax-1.19\ay+3.57\az}، \عددی{0.83\ax-1.81\ay-1.57\az}
\انتہا{سوال}
%================
\ابتدا{سوال}
تین سمتیات \عددی{\kvec{r}_1=2\ax-1\ay+3\az}، \عددی{\kvec{r}_2=-3\ax+4\ay-5\az} اور \عددی{\kvec{r}_3=5\ax-2\ay+3\az} دیے گئے ہیں۔\عددیء{\kvec{r}_1\times \kvec{r}_2} کی سمت میں اکائی سمتیہ حاصل کریں۔ایسی اکائی سمتیہ حاصل کریں جو \عددیء{\kvec{r}_1} اور \عددیء{\kvec{r}_2} دونوں کو عمودی ہو۔سمتیہ \عددیء{\kvec{r}_2-\kvec{r}_1} اور \عددیء{\kvec{r}_2-\kvec{r}_3} دونوں کو عمودی اکائی سمتیہ حاصل کریں۔اس تکون کا رقبہ حاصل کریں جس کے اطراف \عددیء{\kvec{r}_1} اور \عددیء{\kvec{r}_2} ہوں۔اس تکون کا رقبہ حاصل کریں جس کے کونے یہ تین سمتیات دیتے ہیں۔ 

جوابات:\عددی{-0.81\ax+0.16\ay+0.58\az}، \عددی{\mp(-0.81\ax+0.16\ay+0.58\az)}، \عددی{\mp(0.29\ax+0.88\ay+0.37\az)}، \عددی{4.3}، \عددی{13.6}
\انتہا{سوال}
%===============
\ابتدا{سوال}
نقطہ \عددی{N(5,10,4)} پر سمتیات \عددیء{\kvec{R}_{AN}=-3\ax+6\ay+12\az} اور \عددیء{\kvec{R}_{BN}=12\ax+20\ay-5\az} مل کر تکون بناتی ہیں۔تکون کی عمودی اکائی سمتیہ حاصل کریں۔سمتیہ \عددی{\kvec{R}_{BN}} کے عمودی اور تکون کی سطح کے متوازی اکائی سمتیہ حاصل کریں۔تکون کی سطح پر اس اکائی سمتیہ کو حاصل کریں جو نقطہ \عددیء{N} پر تکون کے کونے کو نصف زاویہ میں کاٹے۔

جوابات:\عددی{\mp(-0.83\ax+0.39\ay-0.40\az)}، \عددی{\mp(0.26\ax-0.38\ay-0.89\az)}، \عددی{0.19\ax+0.87\ay+0.45\az}
\انتہا{سوال}
%=============
\ابتدا{سوال}
سمتیہ \عددی{\kvec{M}=(x^2+y^2)^{-1}(x\ax+y\ay)} کو نلکی محدد کے متغیرات میں لکھیں۔نقطہ \عددیء{(5,30^{\circ},6)} پر سمتیہ کی قیمت کارتیسی اور نلکی محدد میں حاصل کریں۔

جوابات:\عددی{\kvec{M}=\tfrac{1}{\rho}\arho}، \عددیء{\kvec{M}=0.41\ax+0.29\ay}، \عددی{\kvec{M}=\tfrac{1}{5}\arho}
\انتہا{سوال}
%===============
\ابتدا{سوال}
نقطہ \عددی{N(\rho=2,\phi=45^{\circ},z=12)} اور \عددیء{P(\rho=5,\phi=-60^{\circ},z=-6)} دئے گئے ہیں۔کارتیسی محدد میں، پہلے نقطے سے دوسرے نقطے کی جانب اکائی سمتیہ حاصل کریں۔اسی اکائی سمتیہ کو پہلے نقطے پر پائے جانے والے نلکی محدد کے متغیرات کی صورت میں لکھیں۔اسی اکائی سمتیہ کو دوسرے نقطے پر پائے جانے والے نلکی محدد کے متغیرات کی صورت میں لکھیں۔

جوابات: \عددی{0.057\ax-0.303\ay-0.951\az}، \عددی{-0.174\arho-0.255\aphi-0.951\az}، \عددی{0.292\arho-0.180\aphi-0.951\az}
\انتہا{سوال}
%================
\ابتدا{سوال}
نقطہ \عددیء{N(\rho=5,\phi=30^{\circ},z=6)} سے نقطہ \عددی{P(\rho=10,\phi=75^{\circ},z=12)} تک سمتیہ کارتیسی محدد میں لکھیں۔اسی سمت میں اکائی سمتیہ بھی لکھیں۔کارتیسی محدد میں دوسرے نقطے سے مرکز تک اکائی سمتیہ لکھیں۔

جوابات:\عددی{-1.74\ax+7.16\ay+6\az}، \عددیء{-0.183\ax-0.618\ay+0.631\az}، \عددی{0.166\ax-0.618\ay-0.768\az}
\انتہا{سوال}
%================
\ابتدا{سوال}
نقطہ \عددی{M(5,-3,2)} سے نقطہ \عددیء{N(10,2,-5)} تک سمتیہ کو نقطہ \عددی{M} پر نلکی محدد کے اکائی سمتیات کی مدد سے لکھیں۔دوسرے نقطے سے پہلے نقطے کی سمت میں اکائی سمتیہ کو دوسرے  نقطے پر نلکی اکائی سمتیات کی مدد سے لکھیں۔دوسرے نقطے سے مرکز تک اکائی سمتیہ دوسرے نقطے کے اکائی سمتیات کی صورت  میں لکھیں۔ 

جوابات: \عددی{-1.71\arho-6.86\aphi+7\az}، \عددی{0.59\arho+0.39\aphi-0.7\az}، \عددی{0.90\arho+0.44\az} 
\انتہا{سوال}
%================
