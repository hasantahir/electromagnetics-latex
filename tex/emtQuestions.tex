\باب{سوالات}
\حصہ{{توانائی باب کے سوالات}}

\ابتدا{سوال}

\انتہا{سوال}
%=====
\ابتدا{سوال}
برقی میدان \عددیء{\kvec{E}=(y+z)\ax+(x+z)\ay+(x+y)\az} میں \عددیء{\SI{-0.1}{\coulomb}} کے چارج کو نقطہ \عددیء{(1,0,2)} سے نقطہ \عددیء{(0,0,2)} اور  یہاں سے نقطہ \عددیء{(0,1,2)} لایا جاتا ہے۔دونوں راستوں کا علیحدہ علیحدہ اور کُل درکار توانائی حاصل کریں۔

جوابات: \عددیء{\SI{0.2}{\joule}}، \عددیء{\SI{-0.2}{\joule}} اور \عددیء{\SI{0}{\joule}}
\انتہا{سوال}

\ابتدا{سوال}
مثال \حوالہ{مثال_توانائی_لامحدود_متوازی_کپیسٹر_کی_توانائی} کے طرز پر \عددیء{L} لمبائی ہم محوری تار میں مخففی توانائی حاصل کریں۔اندرونی تار کا رداس \عددیء{a} جبکہ بیرونی تار کا رداس \عددیء{b} ہے۔

جواب:\عددیء{W=\tfrac{\pi L a^2 \rho_S^2}{\epsilon_0} \ln \frac{b}{a}}
\انتہا{سوال}
%=============================
%=============================
%=============================
\حصہ{کپیسٹر}
%
\ابتدا{سوال}
\عددیء{N(0,0,2)} سے گزرتی \عددیء{y} محدد کے متوازی لکیری چارج کثافت
\begin{align*}
\rho_L=\SI{5}{\nano \coulomb \per \meter}  \quad \quad (-\infty < y < \infty, x=0,z=2)
\end{align*} 
سے \عددیء{M(5,3,1)} پر \عددیء{\kvec{D}} حاصل کریں۔

جواب:\عددیء{\kvec{D}=\tfrac{5\times 10^{-9}(5\ax-1\az)}{2 \pi \times 26}}
\انتہا{سوال}
%==========
\ابتدا{سوال}
لامحدود موصل زمینی سطح \عددیء{z=0}  رکھتے ہوئے  مندرجہ بالا سوال کو دوبارہ حل کریں۔

جواب:\عددیء{\kvec{D}=\tfrac{5\times 10^{-9}(40\ax-112\az)}{2 \pi \times 884}}
\انتہا{سوال}
%====
\ابتدا{سوال}
\عددیء{N(0,0,2)} سے گزرتی \عددیء{y} محدد کے متوازی لکیری چارج کثافت
\begin{align*}
\rho_L=\SI{5}{\nano \coulomb \per \meter}  \quad \quad (-\infty < y < \infty, x=0,z=2)
\end{align*} 
پایا جاتا ہے جبکہ \عددیء{z=0} پر لامحدود موصل زمینی سطح موجود ہے۔سطح کے \عددیء{M(5,3,0)} مقام پر سطحی چارج کثافت حاصل کریں۔

جواب: \عددیء{\SI{-0.1097}{\nano \coulomb \per \meter \squared}}
\انتہا{سوال}
%=======
\ابتدا{سوال}
مشق \حوالہ{مشق_کپیسٹر_نیم_موصل_موصلیت} میں \عددیء{\SI{300}{\kelvin}} درجہ حرارت پر  سلیکان اور جرمینیم کے مستقل دئے گئے ہیں۔اگر سلیکان میں المونیم کا ایک ایٹم فی  \عددیء{\num{1e7}} سلیکان ایٹم  ملاوٹ شامل کی جائے تو سیلکان کی موصلیت کیا ہو گی۔سلیکان کی تعدادی کثافت \عددیء{\num{5e28}} ایٹم فی مربع میٹر ہے۔(ہر ملاوٹی المونیم کا  ایٹم ایک عدد آزاد خول پیدا کرتا ہے جن  کی تعداد مشق میں دئے خالص سلیکان میں آزاد خول کی تعداد سے بہت زیادہ ہوتی ہے لہٰذا ایسی صورت میں موصلیت صرف ملاوٹی ایٹموں کے پیدا کردہ آزاد خول ہی تعین کرتے ہیں۔) 

جواب: \عددیء{\SI{800}{\siemens \per \meter}}
\انتہا{سوال}
%======
\ابتدا{سوال}
صفحہ \حوالہصفحہ{مثال_کپیسٹر_نقطہ_چارج_سے_لامحدود_سطح_میں_پیدا_کثافت} پر مثال \حوالہ{مثال_کپیسٹر_نقطہ_چارج_سے_لامحدود_سطح_میں_پیدا_کثافت} میں لامحدود موصل سطح \عددیء{z=0} میں \عددیء{(0,0,z)} پر پائے جانے والے نقطہ چارج \عددیء{Q} سے پیدا سطحی چارج کثافت \عددیء{\rho_S} حاصل کیا گیا۔موصل سطح میں پائے جانے والا کل چارج سطحی تکمل سے حاصل کریں۔

جواب: \عددیء{-Q} 
\انتہا{سوال}
%========================
\ابتدا{سوال}
صفحہ \حوالہصفحہ{مساوات_کپیسٹر_موصل_آزاد_چارج_کثافت} پر تانبے کے ایک مربع میٹر میں کل آزاد چارج مساوات \حوالہ{مساوات_کپیسٹر_موصل_آزاد_چارج_کثافت} میں حاصل کیا گیا۔ایک ایمپئیر کی برقی رو کتنے وقت میں اتنے چارج کا اخراج کرے گا۔

جواب: چار سو اکتیس \عددیء{(431)} سال۔ 
\انتہا{سوال}
%=========================
