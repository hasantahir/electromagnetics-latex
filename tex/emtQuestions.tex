\باب{سوالات}
\section*{سمتیات}
\ابتدا{سوال}
سمتیہ \عددی{\kvec{A}=-2\ax+1\ay+7\az} اور \عددی{\kvec{B}=3\ax+5\ay-2\az} ہیں۔مندرجہ ذیل حاصل کریں: (الف) \عددی{2\kvec{A}-3\kvec{B}} اور اسی کی سمت میں اکائی سمتیہ؛ (ب) \عددی{2 \kvec{A}-5\kvec{B}+3\ax}؛ (پ) \عددی{\abs{3\kvec{A}}\abs{2\kvec{B}}(\kvec{B}-\kvec{A})}

جوابات: \عددی{-13\ax-13\ay+8\az}، \عددی{-0.648\ax-0.648\ay-0.399\az}، \عددی{28.3}، \عددی{1359\ax+1087\ay+1359\az}
\انتہا{سوال}

%=============================
\ابتدا{سوال}
نقطہ \عددی{A(1,-2,3)}، \عددی{B(3,-1,2)} اور \عددی{C(7,5,-4)} دیے گئے ہیں۔(الف) محدد کے مرکز سے \عددی{A} تک سمتیہ لکھیں؛ (ب) مرکز سے لکیر \عددی{AB} کے وسط تک سمتیہ لکھیں؛ (پ) اسی سمت میں اکائی سمتیہ لکھیں؛ (ت) تکون \عددی{ABC} کا احاطہ دریافت کریں۔

جوابات: \عددی{\ax-2\ay+3\az}، \عددی{2\ax-1.5\ay+2.5\az}، \عددی{0.566\ax-0.424\ay-0.707\az}، \عددی{23.4} 
\انتہا{سوال}
%==========================
\ابتدا{سوال}
مرکز سے نقطہ \عددیء{A} تک سمتیہ \عددیء{2\ax+\ay+3\az} ہے جبکہ مرکز سے \عددیء{\tfrac{2}{3}\ax-\tfrac{2}{3}\ay+\tfrac{1}{3}\az} اکائی سمتیہ کی سمت میں نقطہ \عددیء{B} پایا جاتا ہے۔دونوں نقطوں کے درمیان \عددیء{4} فاصلہ  ہونے کی صورت میں نقطہ \عددیء{B} دریافت کریں۔

جوابات:\عددیء{(2.57,-2.57,1.28)}
\انتہا{سوال}
%===============================
\ابتدا{سوال}
سمتی میدان \عددیء{\kvec{M}=(x+y^2)\ax+2(xy+3)\ay+4z^2\az} دیا گیا ہے۔نقطہ \عددیء{A(2,-3,1)} پر اس میدان کی قیمت حاصل کریں۔اسی نقطے پر میدان کی سمت میں اکائی سمتیہ دریافت کریں۔ایسی سطح جس پر \عددی{\abs{\kvec{M}}=5} ہو کی مساوات حاصل کریں۔اس سطح پر \عددیء{y=2} اور \عددیء{z=-1} ہونے کی صورت میں حاصل لکیر کی مساوات حاصل کریں۔

جوابات:\عددی{\kvec{M}=11\ax-6\ay+4\az}، \عددی{(0.836\ax-0.456\ay+0.304\az)}، \\ \عددی{x^2+y^2+2xy^2+4x^2y^2+24xy+16z^4-11=0}، \عددی{17x^2+56x+9=0}
\انتہا{سوال}
%===============================
\ابتدا{سوال}
سمتی میدان \عددی{\kvec{B}=2x^2\ax-3y(x+2z)\ay+5\az} اور \عددیء{\kvec{M}=(x+y+z)\ax+\tfrac{y}{x}\ay+xy\az} دیے گئے ہیں۔نقطہ \عددی{N(2,-3,-1)} پر \عددی{\kvec{B}} اور \عددیء{\kvec{M}} حاصل کریں۔اسی نقطے پر سمتیہ \عددیء{2\kvec{B}-\kvec{M}} کی سمت میں اکائی سمتیہ حاصل کریں۔

جوابات:\عددی{\kvec{B}=8\ax+5\az}، \عددی{\kvec{M}=-2\ax-1.5\ay-2\az}، \عددی{0.830\ax+0.069\ay+0.553\az}
\انتہا{سوال}
%================
\ابتدا{سوال}
 نقطہ \عددیء{N(2,-3,7))} پر میدان \عددیء{\kvec{M}=\tfrac{16}{x^2+y^2}(x\ax+y\ay)} کی سمت میں اکائی سمتیہ \عددی{\kvec{a}_M} دریافت کریں۔نقطہ \عددیء{N} پر \عددیء{\ax} اور \عددیء{\kvec{M}} کے درمیان زاویہ حاصل کریں۔اسی طرح نقطہ \عددیء{N} پر \عددیء{\ay} اور \عددیء{\kvec{M}} کے درمیان زاویہ حاصل کریں۔

جوابات:\عددی{\kvec{a}_M=0.555\ax-0.832\ay}، \عددی{56.3^{\circ}}، \عددی{33.7^{\circ}}
\انتہا{سوال}
%========================
\ابتدا{سوال}
میدان \عددیء{\kvec{M}=\tfrac{16}{x^2+y^2}(x\ax+y\ay)} کا مندرجہ ذیل دو درجی تکمل \عددی{y=3} سطح پر حاصل کریں۔
\begin{align*}
\int_{0}^{3} \int_{0}^{2} \kvec{M} \dif x \dif z\cdot \ax 
\end{align*}

جواب: \عددیء{24 \ln \frac{13}{9}}
\انتہا{سوال}
%========================
\ابتدا{سوال}
غیر سمتی ضرب استعمال کرتے ہوئے تکون \عددیء{ABC} میں زاویہ \عددی{A} اور \عددی{C} حاصل کریں۔تکون کے کونے \عددی{A(3,1,2)}، \عددی{B(4,6,2)} اور
 \عددی{C(1,4,-2)} ہیں۔

جوابات:\عددی{61.74^{\circ}} ، \عددیء{56.51^{\circ}}
\انتہا{سوال}
%===================
\ابتدا{سوال}
نقطے \عددیء{A(4,1,2)}، \عددیء{B(-2,4,3)} اور \عددیء{C(2,3,-1)} دیے گئے ہیں۔سمتیہ \عددی{\kvec{R}_{BA}} اور \عددی{\kvec{R}_{CA}} حاصل کریں۔دوسری سمتیے  پر  پہلی سمتیہ کے \اصطلاح{عمودی سایے}\فرہنگ{عمودی سایے}\فرہنگ{سایہ!عمودی}\حاشیہب{projection}\فرہنگ{projection} کی لمبائی دریافت کریں۔لکیر \عددیء{AB} کے درمیانے نقطے سے لکیر \عددی{AC} کے درمیانے نقطے تک سیدھا سمتیہ حاصل کریں۔

جوابات: \عددی{-6\ax+3\ay+\az}، \عددی{-2\ax+2\ay-3\az}، \عددی{4.12}، \عددی{2\ax-0.5\ay-2\az}
\انتہا{سوال}
%==================
