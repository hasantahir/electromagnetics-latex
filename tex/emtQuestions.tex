\حصہ{توانائی باب کے سوالات}


\ابتدا{سوال}

\انتہا{سوال}
%=====
\ابتدا{سوال}
برقی میدان \عددیء{\kvec{E}=(y+z)\ax+(x+z)\ay+(x+y)\az} میں \عددیء{\SI{-0.1}{\coulomb}} کے چارج کو نقطہ \عددیء{(1,0,2)} سے نقطہ \عددیء{(0,0,2)} اور  یہاں سے نقطہ \عددیء{(0,1,2)} لایا جاتا ہے۔دونوں راستوں کا علیحدہ علیحدہ اور کُل درکار توانائی حاصل کریں۔

جوابات: \عددیء{\SI{0.2}{\joule}}، \عددیء{\SI{-0.2}{\joule}} اور \عددیء{\SI{0}{\joule}}
\انتہا{سوال}

\ابتدا{سوال}
مثال \حوالہ{مثال_توانائی_لامحدود_متوازی_کپیسٹر_کی_توانائی} کے طرز پر \عددیء{L} لمبائی ہم محوری تار میں مخففی توانائی حاصل کریں۔اندرونی تار کا رداس \عددیء{a} جبکہ بیرونی تار کا رداس \عددیء{b} ہے۔

جواب:\عددیء{W=\tfrac{\pi L a^2 \rho_S^2}{\epsilon_0} \ln \frac{b}{a}}
\انتہا{سوال}
