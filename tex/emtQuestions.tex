\باب{سوالات}
\حصہ{{توانائی باب کے سوالات}}

\ابتدا{سوال}

\انتہا{سوال}
%=====
\ابتدا{سوال}
برقی میدان \عددیء{\kvec{E}=(y+z)\ax+(x+z)\ay+(x+y)\az} میں \عددیء{\SI{-0.1}{\coulomb}} کے چارج کو نقطہ \عددیء{(1,0,2)} سے نقطہ \عددیء{(0,0,2)} اور  یہاں سے نقطہ \عددیء{(0,1,2)} لایا جاتا ہے۔دونوں راستوں کا علیحدہ علیحدہ اور کُل درکار توانائی حاصل کریں۔

جوابات: \عددیء{\SI{0.2}{\joule}}، \عددیء{\SI{-0.2}{\joule}} اور \عددیء{\SI{0}{\joule}}
\انتہا{سوال}

\ابتدا{سوال}
مثال \حوالہ{مثال_توانائی_لامحدود_متوازی_کپیسٹر_کی_توانائی} کے طرز پر \عددیء{L} لمبائی ہم محوری تار میں مخففی توانائی حاصل کریں۔اندرونی تار کا رداس \عددیء{a} جبکہ بیرونی تار کا رداس \عددیء{b} ہے۔

جواب:\عددیء{W=\tfrac{\pi L a^2 \rho_S^2}{\epsilon_0} \ln \frac{b}{a}}
\انتہا{سوال}
%=============================
%=============================
%=============================
\حصہ{کپیسٹر}
%
\ابتدا{سوال}
\عددیء{N(0,0,2)} سے گزرتی \عددیء{y} محدد کے متوازی لکیری چارج کثافت
\begin{align*}
\rho_L=\SI{5}{\nano \coulomb \per \meter}  \quad \quad (-\infty < y < \infty, x=0,z=2)
\end{align*} 
سے \عددیء{M(5,3,1)} پر \عددیء{\kvec{D}} حاصل کریں۔

جواب:\عددیء{\kvec{D}=\tfrac{5\times 10^{-9}(5\ax-1\az)}{2 \pi \times 26}}
\انتہا{سوال}
%==========
\ابتدا{سوال}
لامحدود موصل زمینی سطح \عددیء{z=0}  رکھتے ہوئے  مندرجہ بالا سوال کو دوبارہ حل کریں۔

جواب:\عددیء{\kvec{D}=\tfrac{5\times 10^{-9}(40\ax-112\az)}{2 \pi \times 884}}
\انتہا{سوال}
%====
\ابتدا{سوال}
\عددیء{N(0,0,2)} سے گزرتی \عددیء{y} محدد کے متوازی لکیری چارج کثافت
\begin{align*}
\rho_L=\SI{5}{\nano \coulomb \per \meter}  \quad \quad (-\infty < y < \infty, x=0,z=2)
\end{align*} 
پایا جاتا ہے جبکہ \عددیء{z=0} پر لامحدود موصل زمینی سطح موجود ہے۔سطح کے \عددیء{M(5,3,0)} مقام پر سطحی چارج کثافت حاصل کریں۔

جواب: \عددیء{\SI{-0.1097}{\nano \coulomb \per \meter \squared}}
\انتہا{سوال}
%=======
\ابتدا{سوال}
مشق \حوالہ{مشق_کپیسٹر_نیم_موصل_موصلیت} میں \عددیء{\SI{300}{\kelvin}} درجہ حرارت پر  سلیکان اور جرمینیم کے مستقل دئے گئے ہیں۔اگر سلیکان میں المونیم کا ایک ایٹم فی  \عددیء{\num{1e7}} سلیکان ایٹم  ملاوٹ شامل کی جائے تو سیلکان کی موصلیت کیا ہو گی۔سلیکان کی تعدادی کثافت \عددیء{\num{5e28}} ایٹم فی مربع میٹر ہے۔(ہر ملاوٹی المونیم کا  ایٹم ایک عدد آزاد خول پیدا کرتا ہے جن  کی تعداد مشق میں دئے خالص سلیکان میں آزاد خول کی تعداد سے بہت زیادہ ہوتی ہے لہٰذا ایسی صورت میں موصلیت صرف ملاوٹی ایٹموں کے پیدا کردہ آزاد خول ہی تعین کرتے ہیں۔) 

جواب: \عددیء{\SI{800}{\siemens \per \meter}}
\انتہا{سوال}
%======
\ابتدا{سوال}
صفحہ \حوالہصفحہ{مثال_کپیسٹر_نقطہ_چارج_سے_لامحدود_سطح_میں_پیدا_کثافت} پر مثال \حوالہ{مثال_کپیسٹر_نقطہ_چارج_سے_لامحدود_سطح_میں_پیدا_کثافت} میں لامحدود موصل سطح \عددیء{z=0} میں \عددیء{(0,0,z)} پر پائے جانے والے نقطہ چارج \عددیء{Q} سے پیدا سطحی چارج کثافت \عددیء{\rho_S} حاصل کیا گیا۔موصل سطح میں پائے جانے والا کل چارج سطحی تکمل سے حاصل کریں۔

جواب: \عددیء{-Q} 
\انتہا{سوال}
%========================
\ابتدا{سوال}
صفحہ \حوالہصفحہ{مساوات_کپیسٹر_موصل_آزاد_چارج_کثافت} پر تانبے کے ایک مربع میٹر میں کل آزاد چارج مساوات \حوالہ{مساوات_کپیسٹر_موصل_آزاد_چارج_کثافت} میں حاصل کیا گیا۔ایک ایمپئیر کی برقی رو کتنے وقت میں اتنے چارج کا اخراج کرے گا۔

جواب: چار سو اکتیس \عددیء{(431)} سال۔ 
\انتہا{سوال}
%=========================

\ابتدا{سوال}
مساوات \حوالہ{مساوات_کپیسٹر_نلکی_زمین_کپیسٹنس} میں  \عددیء{\ln \frac{h+\sqrt{h^2-b^2}}{b}=\cosh^{-1} \frac{h}{b}} لکھا گیا ہے۔اسے ثابت کریں۔
\انتہا{سوال}
%===========

\ابتدا{سوال}
پانچ میٹر رداس کی موصل نلکی کا محور برقی زمین سے تیرا میٹر پر ہے۔نلکی پر ایک سو وولٹ کا برقی دباو ہے۔
\begin{itemize}
\item
ایسی لکیری چارج کثافت کا زمین سے فاصلا اور اس کا \عددیء{\rho_L} حاصل کریں جو ایسی ہم قوہ سطح  پیدا کرے۔
\item
موصل نلکی سے پیدا پچاس وولٹ کے ہم قوہ سطح کا رداس اور اس کے محور کا زمین سے فاصلا دریافت کریں۔
\item
نلکی پر زمین کے قریب اور اس سے دور سطحی چارج کثافت حاصل کریں۔
\end{itemize} 

جوابات:\عددیء{\SI{12}{\meter}}، \عددیء{\SI{3.46}{\nano \coulomb \per \meter}}، \عددیء{\SI{13.4}{\meter}}، \عددیء{\SI{18}{\meter}}، \عددیء{\SI{1.65}{\pico \farad \per \meter \squared}} اور \عددیء{\SI{0.73}{\pico \farad \per \meter \squared}}

\انتہا{سوال}
%=============
\حصہ{لاپلاس}
\ابتدا{سوال}
صفحہ \حوالہصفحہ{مساوات_لاپلاس_عمومی_لاپلاسی} پر مساوات \حوالہ{مساوات_لاپلاس_عمومی_لاپلاسی} عمومی محدد میں لاپلاسی دیتا ہے۔اس مساوات کو حاصل کریں۔
\انتہا{سوال}
%=============
\ابتدا{سوال}
مثال \حوالہ{مثال_لاپلاس_رداسی_چادروں_کا_کپیسٹر} کو حتمی نتیجے تک پہنچاتے ہوئے اس  کا کپیسٹنس حاصل کریں۔
\انتہا{سوال}
%====================
\ابتدا{سوال}
مثال \حوالہ{مثال_لاپلاس_کروی_رداسی_لاپلاسی} میں دیے مساوات \حوالہ{مساوات_لاپلاس_کروی_رداسی_لاپلاسی_دباو} اور مساوات \حوالہ{مساوات_لاپلاس_کروی_رداسی_لاپلاسی_کپیسٹنس} حاصل کریں۔
\انتہا{سوال}
%=====================
\ابتدا{سوال}
مساوات \حوالہ{مساوات_لاپلاس_مخروطی_حل} کے تکمل کو حل کریں۔
\انتہا{سوال}
%==================
\ابتدا{سوال}
مساوات \حوالہ{مساوات_لاپلاس_مخروطی_حل_ب} حاصل کریں۔
\انتہا{سوال}
%========================
\ابتدا{سوال}
مساوات \حوالہ{مساوات_لاپلاس_مخروطی_کپیسٹنس} حل کریں۔
\انتہا{سوال}
%==========================
\ابتدا{سوال}
مساوات \حوالہ{مساوات_لاپلاس_تفرقی_اجزاء_مستقل_کے_برابر_ب} کے دوسرے جزو کا حل طاقتی سلسلے کے طریقے سے حاصل کریں۔ثابت کریں کہ اس حل کو مساوات \حوالہ{مساوات_لاپلاس_طاقتی_حل_دو_درجی_کارتیسی} کی شکل میں لکھا جا سکتا ہے۔
\انتہا{سوال}
%===============
\ابتدا{سوال}
دہرانے کے طریقے میں اشاریہ کے نشان کے بعد دو ہندسوں تک درستگی استعمال کرتے ہوئے شکل \حوالہ{شکل_لاپلاس_دہرانے_کا_طریقہ} میں دئے تمام نقطوں پر برقی دباو چار مرتبہ دہرانے سے حاصل کریں۔ڈبے کے وسط میں برقی دباو کیا حاصل ہوتی ہے۔

جواب: \عددیء{\SI{22.49}{\volt}}
\انتہا{سوال}
%===================
%====================
\حصہ{بایوٹ-سیوارٹ}

\ابتدا{سوال}
مساوات \حوالہ{مساوات_مقناطیسی_محدود_تار_کی_مقناطیس} حاصل کریں۔
\انتہا{سوال}
%===============
\ابتدا{سوال}
شکل \حوالہ{شکل_مقناطیسی_لامحدود_سطحی_برقی_رو} کے لامحدود سطح سے پیدا مقناطیسی میدان بایوٹ-سیوارٹ کے قانون کی مدد سے حاصل کریں۔
\انتہا{سوال}
%=============
\ابتدا{سوال}
مساوات \حوالہ{مساوات_مقناطیسی_راہ_پہلے_حصے_پر_شدت_دوبارہ} حاصل کرنے کے طرز پر شکل \حوالہ{شکل-مقناطیسی_گردش_تعریف} میں \عددیء{3} تا \عددیء{4} پر \عددیء{H_{y34}} حاصل کریں۔

جواب:شرح \عددیء{\tfrac{\partial H_y}{\partial x}} ہے۔یوں \عددیء{-\tfrac{\Delta x}{2}} تبدیلی سے  \عددیء{\tfrac{\partial H_y}{\partial x}(-\frac{\Delta x}{2})} تبدیلی رو نما ہو گی اور یوں نئی قیمت 
{\عددیء{H_{y0}-\tfrac{\partial H_y}{\partial x}\tfrac{\Delta x}{2}}} ہو گی۔

\انتہا{سوال}
%===========
\ابتدا{سوال}
عمومی محدد میں حاصل کردہ گردش کی مساوات سے کارتیسی محدد میں گردش کی مساوات حاصل کریں۔
\انتہا{سوال}
%==============================

\ابتدا{سوال}
عمومی محدد میں حاصل کردہ گردش کی مساوات سے نلکی محدد میں گردش کی مساوات حاصل کریں۔
\انتہا{سوال}
